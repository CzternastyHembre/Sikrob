\documentclass[11pt]{article}

    \usepackage[breakable]{tcolorbox}
    \usepackage{parskip} % Stop auto-indenting (to mimic markdown behaviour)
    
    \usepackage{iftex}
    \ifPDFTeX
    	\usepackage[T1]{fontenc}
    	\usepackage{mathpazo}
    \else
    	\usepackage{fontspec}
    \fi

    % Basic figure setup, for now with no caption control since it's done
    % automatically by Pandoc (which extracts ![](path) syntax from Markdown).
    \usepackage{graphicx}
    % Maintain compatibility with old templates. Remove in nbconvert 6.0
    \let\Oldincludegraphics\includegraphics
    % Ensure that by default, figures have no caption (until we provide a
    % proper Figure object with a Caption API and a way to capture that
    % in the conversion process - todo).
    \usepackage{caption}
    \DeclareCaptionFormat{nocaption}{}
    \captionsetup{format=nocaption,aboveskip=0pt,belowskip=0pt}

    \usepackage{float}
    \floatplacement{figure}{H} % forces figures to be placed at the correct location
    \usepackage{xcolor} % Allow colors to be defined
    \usepackage{enumerate} % Needed for markdown enumerations to work
    \usepackage{geometry} % Used to adjust the document margins
    \usepackage{amsmath} % Equations
    \usepackage{amssymb} % Equations
    \usepackage{textcomp} % defines textquotesingle
    % Hack from http://tex.stackexchange.com/a/47451/13684:
    \AtBeginDocument{%
        \def\PYZsq{\textquotesingle}% Upright quotes in Pygmentized code
    }
    \usepackage{upquote} % Upright quotes for verbatim code
    \usepackage{eurosym} % defines \euro
    \usepackage[mathletters]{ucs} % Extended unicode (utf-8) support
    \usepackage{fancyvrb} % verbatim replacement that allows latex
    \usepackage{grffile} % extends the file name processing of package graphics 
                         % to support a larger range
    \makeatletter % fix for old versions of grffile with XeLaTeX
    \@ifpackagelater{grffile}{2019/11/01}
    {
      % Do nothing on new versions
    }
    {
      \def\Gread@@xetex#1{%
        \IfFileExists{"\Gin@base".bb}%
        {\Gread@eps{\Gin@base.bb}}%
        {\Gread@@xetex@aux#1}%
      }
    }
    \makeatother
    \usepackage[Export]{adjustbox} % Used to constrain images to a maximum size
    \adjustboxset{max size={0.9\linewidth}{0.9\paperheight}}

    % The hyperref package gives us a pdf with properly built
    % internal navigation ('pdf bookmarks' for the table of contents,
    % internal cross-reference links, web links for URLs, etc.)
    \usepackage{hyperref}
    % The default LaTeX title has an obnoxious amount of whitespace. By default,
    % titling removes some of it. It also provides customization options.
    \usepackage{titling}
    \usepackage{longtable} % longtable support required by pandoc >1.10
    \usepackage{booktabs}  % table support for pandoc > 1.12.2
    \usepackage[inline]{enumitem} % IRkernel/repr support (it uses the enumerate* environment)
    \usepackage[normalem]{ulem} % ulem is needed to support strikethroughs (\sout)
                                % normalem makes italics be italics, not underlines
    \usepackage{mathrsfs}
    

    
    % Colors for the hyperref package
    \definecolor{urlcolor}{rgb}{0,.145,.698}
    \definecolor{linkcolor}{rgb}{.71,0.21,0.01}
    \definecolor{citecolor}{rgb}{.12,.54,.11}

    % ANSI colors
    \definecolor{ansi-black}{HTML}{3E424D}
    \definecolor{ansi-black-intense}{HTML}{282C36}
    \definecolor{ansi-red}{HTML}{E75C58}
    \definecolor{ansi-red-intense}{HTML}{B22B31}
    \definecolor{ansi-green}{HTML}{00A250}
    \definecolor{ansi-green-intense}{HTML}{007427}
    \definecolor{ansi-yellow}{HTML}{DDB62B}
    \definecolor{ansi-yellow-intense}{HTML}{B27D12}
    \definecolor{ansi-blue}{HTML}{208FFB}
    \definecolor{ansi-blue-intense}{HTML}{0065CA}
    \definecolor{ansi-magenta}{HTML}{D160C4}
    \definecolor{ansi-magenta-intense}{HTML}{A03196}
    \definecolor{ansi-cyan}{HTML}{60C6C8}
    \definecolor{ansi-cyan-intense}{HTML}{258F8F}
    \definecolor{ansi-white}{HTML}{C5C1B4}
    \definecolor{ansi-white-intense}{HTML}{A1A6B2}
    \definecolor{ansi-default-inverse-fg}{HTML}{FFFFFF}
    \definecolor{ansi-default-inverse-bg}{HTML}{000000}

    % common color for the border for error outputs.
    \definecolor{outerrorbackground}{HTML}{FFDFDF}

    % commands and environments needed by pandoc snippets
    % extracted from the output of `pandoc -s`
    \providecommand{\tightlist}{%
      \setlength{\itemsep}{0pt}\setlength{\parskip}{0pt}}
    \DefineVerbatimEnvironment{Highlighting}{Verbatim}{commandchars=\\\{\}}
    % Add ',fontsize=\small' for more characters per line
    \newenvironment{Shaded}{}{}
    \newcommand{\KeywordTok}[1]{\textcolor[rgb]{0.00,0.44,0.13}{\textbf{{#1}}}}
    \newcommand{\DataTypeTok}[1]{\textcolor[rgb]{0.56,0.13,0.00}{{#1}}}
    \newcommand{\DecValTok}[1]{\textcolor[rgb]{0.25,0.63,0.44}{{#1}}}
    \newcommand{\BaseNTok}[1]{\textcolor[rgb]{0.25,0.63,0.44}{{#1}}}
    \newcommand{\FloatTok}[1]{\textcolor[rgb]{0.25,0.63,0.44}{{#1}}}
    \newcommand{\CharTok}[1]{\textcolor[rgb]{0.25,0.44,0.63}{{#1}}}
    \newcommand{\StringTok}[1]{\textcolor[rgb]{0.25,0.44,0.63}{{#1}}}
    \newcommand{\CommentTok}[1]{\textcolor[rgb]{0.38,0.63,0.69}{\textit{{#1}}}}
    \newcommand{\OtherTok}[1]{\textcolor[rgb]{0.00,0.44,0.13}{{#1}}}
    \newcommand{\AlertTok}[1]{\textcolor[rgb]{1.00,0.00,0.00}{\textbf{{#1}}}}
    \newcommand{\FunctionTok}[1]{\textcolor[rgb]{0.02,0.16,0.49}{{#1}}}
    \newcommand{\RegionMarkerTok}[1]{{#1}}
    \newcommand{\ErrorTok}[1]{\textcolor[rgb]{1.00,0.00,0.00}{\textbf{{#1}}}}
    \newcommand{\NormalTok}[1]{{#1}}
    
    % Additional commands for more recent versions of Pandoc
    \newcommand{\ConstantTok}[1]{\textcolor[rgb]{0.53,0.00,0.00}{{#1}}}
    \newcommand{\SpecialCharTok}[1]{\textcolor[rgb]{0.25,0.44,0.63}{{#1}}}
    \newcommand{\VerbatimStringTok}[1]{\textcolor[rgb]{0.25,0.44,0.63}{{#1}}}
    \newcommand{\SpecialStringTok}[1]{\textcolor[rgb]{0.73,0.40,0.53}{{#1}}}
    \newcommand{\ImportTok}[1]{{#1}}
    \newcommand{\DocumentationTok}[1]{\textcolor[rgb]{0.73,0.13,0.13}{\textit{{#1}}}}
    \newcommand{\AnnotationTok}[1]{\textcolor[rgb]{0.38,0.63,0.69}{\textbf{\textit{{#1}}}}}
    \newcommand{\CommentVarTok}[1]{\textcolor[rgb]{0.38,0.63,0.69}{\textbf{\textit{{#1}}}}}
    \newcommand{\VariableTok}[1]{\textcolor[rgb]{0.10,0.09,0.49}{{#1}}}
    \newcommand{\ControlFlowTok}[1]{\textcolor[rgb]{0.00,0.44,0.13}{\textbf{{#1}}}}
    \newcommand{\OperatorTok}[1]{\textcolor[rgb]{0.40,0.40,0.40}{{#1}}}
    \newcommand{\BuiltInTok}[1]{{#1}}
    \newcommand{\ExtensionTok}[1]{{#1}}
    \newcommand{\PreprocessorTok}[1]{\textcolor[rgb]{0.74,0.48,0.00}{{#1}}}
    \newcommand{\AttributeTok}[1]{\textcolor[rgb]{0.49,0.56,0.16}{{#1}}}
    \newcommand{\InformationTok}[1]{\textcolor[rgb]{0.38,0.63,0.69}{\textbf{\textit{{#1}}}}}
    \newcommand{\WarningTok}[1]{\textcolor[rgb]{0.38,0.63,0.69}{\textbf{\textit{{#1}}}}}
    
    
    % Define a nice break command that doesn't care if a line doesn't already
    % exist.
    \def\br{\hspace*{\fill} \\* }
    % Math Jax compatibility definitions
    \def\gt{>}
    \def\lt{<}
    \let\Oldtex\TeX
    \let\Oldlatex\LaTeX
    \renewcommand{\TeX}{\textrm{\Oldtex}}
    \renewcommand{\LaTeX}{\textrm{\Oldlatex}}
    % Document parameters
    % Document title
    \title{Oving3-revidert-v1}
    
    
    
    
    
% Pygments definitions
\makeatletter
\def\PY@reset{\let\PY@it=\relax \let\PY@bf=\relax%
    \let\PY@ul=\relax \let\PY@tc=\relax%
    \let\PY@bc=\relax \let\PY@ff=\relax}
\def\PY@tok#1{\csname PY@tok@#1\endcsname}
\def\PY@toks#1+{\ifx\relax#1\empty\else%
    \PY@tok{#1}\expandafter\PY@toks\fi}
\def\PY@do#1{\PY@bc{\PY@tc{\PY@ul{%
    \PY@it{\PY@bf{\PY@ff{#1}}}}}}}
\def\PY#1#2{\PY@reset\PY@toks#1+\relax+\PY@do{#2}}

\@namedef{PY@tok@w}{\def\PY@tc##1{\textcolor[rgb]{0.73,0.73,0.73}{##1}}}
\@namedef{PY@tok@c}{\let\PY@it=\textit\def\PY@tc##1{\textcolor[rgb]{0.25,0.50,0.50}{##1}}}
\@namedef{PY@tok@cp}{\def\PY@tc##1{\textcolor[rgb]{0.74,0.48,0.00}{##1}}}
\@namedef{PY@tok@k}{\let\PY@bf=\textbf\def\PY@tc##1{\textcolor[rgb]{0.00,0.50,0.00}{##1}}}
\@namedef{PY@tok@kp}{\def\PY@tc##1{\textcolor[rgb]{0.00,0.50,0.00}{##1}}}
\@namedef{PY@tok@kt}{\def\PY@tc##1{\textcolor[rgb]{0.69,0.00,0.25}{##1}}}
\@namedef{PY@tok@o}{\def\PY@tc##1{\textcolor[rgb]{0.40,0.40,0.40}{##1}}}
\@namedef{PY@tok@ow}{\let\PY@bf=\textbf\def\PY@tc##1{\textcolor[rgb]{0.67,0.13,1.00}{##1}}}
\@namedef{PY@tok@nb}{\def\PY@tc##1{\textcolor[rgb]{0.00,0.50,0.00}{##1}}}
\@namedef{PY@tok@nf}{\def\PY@tc##1{\textcolor[rgb]{0.00,0.00,1.00}{##1}}}
\@namedef{PY@tok@nc}{\let\PY@bf=\textbf\def\PY@tc##1{\textcolor[rgb]{0.00,0.00,1.00}{##1}}}
\@namedef{PY@tok@nn}{\let\PY@bf=\textbf\def\PY@tc##1{\textcolor[rgb]{0.00,0.00,1.00}{##1}}}
\@namedef{PY@tok@ne}{\let\PY@bf=\textbf\def\PY@tc##1{\textcolor[rgb]{0.82,0.25,0.23}{##1}}}
\@namedef{PY@tok@nv}{\def\PY@tc##1{\textcolor[rgb]{0.10,0.09,0.49}{##1}}}
\@namedef{PY@tok@no}{\def\PY@tc##1{\textcolor[rgb]{0.53,0.00,0.00}{##1}}}
\@namedef{PY@tok@nl}{\def\PY@tc##1{\textcolor[rgb]{0.63,0.63,0.00}{##1}}}
\@namedef{PY@tok@ni}{\let\PY@bf=\textbf\def\PY@tc##1{\textcolor[rgb]{0.60,0.60,0.60}{##1}}}
\@namedef{PY@tok@na}{\def\PY@tc##1{\textcolor[rgb]{0.49,0.56,0.16}{##1}}}
\@namedef{PY@tok@nt}{\let\PY@bf=\textbf\def\PY@tc##1{\textcolor[rgb]{0.00,0.50,0.00}{##1}}}
\@namedef{PY@tok@nd}{\def\PY@tc##1{\textcolor[rgb]{0.67,0.13,1.00}{##1}}}
\@namedef{PY@tok@s}{\def\PY@tc##1{\textcolor[rgb]{0.73,0.13,0.13}{##1}}}
\@namedef{PY@tok@sd}{\let\PY@it=\textit\def\PY@tc##1{\textcolor[rgb]{0.73,0.13,0.13}{##1}}}
\@namedef{PY@tok@si}{\let\PY@bf=\textbf\def\PY@tc##1{\textcolor[rgb]{0.73,0.40,0.53}{##1}}}
\@namedef{PY@tok@se}{\let\PY@bf=\textbf\def\PY@tc##1{\textcolor[rgb]{0.73,0.40,0.13}{##1}}}
\@namedef{PY@tok@sr}{\def\PY@tc##1{\textcolor[rgb]{0.73,0.40,0.53}{##1}}}
\@namedef{PY@tok@ss}{\def\PY@tc##1{\textcolor[rgb]{0.10,0.09,0.49}{##1}}}
\@namedef{PY@tok@sx}{\def\PY@tc##1{\textcolor[rgb]{0.00,0.50,0.00}{##1}}}
\@namedef{PY@tok@m}{\def\PY@tc##1{\textcolor[rgb]{0.40,0.40,0.40}{##1}}}
\@namedef{PY@tok@gh}{\let\PY@bf=\textbf\def\PY@tc##1{\textcolor[rgb]{0.00,0.00,0.50}{##1}}}
\@namedef{PY@tok@gu}{\let\PY@bf=\textbf\def\PY@tc##1{\textcolor[rgb]{0.50,0.00,0.50}{##1}}}
\@namedef{PY@tok@gd}{\def\PY@tc##1{\textcolor[rgb]{0.63,0.00,0.00}{##1}}}
\@namedef{PY@tok@gi}{\def\PY@tc##1{\textcolor[rgb]{0.00,0.63,0.00}{##1}}}
\@namedef{PY@tok@gr}{\def\PY@tc##1{\textcolor[rgb]{1.00,0.00,0.00}{##1}}}
\@namedef{PY@tok@ge}{\let\PY@it=\textit}
\@namedef{PY@tok@gs}{\let\PY@bf=\textbf}
\@namedef{PY@tok@gp}{\let\PY@bf=\textbf\def\PY@tc##1{\textcolor[rgb]{0.00,0.00,0.50}{##1}}}
\@namedef{PY@tok@go}{\def\PY@tc##1{\textcolor[rgb]{0.53,0.53,0.53}{##1}}}
\@namedef{PY@tok@gt}{\def\PY@tc##1{\textcolor[rgb]{0.00,0.27,0.87}{##1}}}
\@namedef{PY@tok@err}{\def\PY@bc##1{{\setlength{\fboxsep}{\string -\fboxrule}\fcolorbox[rgb]{1.00,0.00,0.00}{1,1,1}{\strut ##1}}}}
\@namedef{PY@tok@kc}{\let\PY@bf=\textbf\def\PY@tc##1{\textcolor[rgb]{0.00,0.50,0.00}{##1}}}
\@namedef{PY@tok@kd}{\let\PY@bf=\textbf\def\PY@tc##1{\textcolor[rgb]{0.00,0.50,0.00}{##1}}}
\@namedef{PY@tok@kn}{\let\PY@bf=\textbf\def\PY@tc##1{\textcolor[rgb]{0.00,0.50,0.00}{##1}}}
\@namedef{PY@tok@kr}{\let\PY@bf=\textbf\def\PY@tc##1{\textcolor[rgb]{0.00,0.50,0.00}{##1}}}
\@namedef{PY@tok@bp}{\def\PY@tc##1{\textcolor[rgb]{0.00,0.50,0.00}{##1}}}
\@namedef{PY@tok@fm}{\def\PY@tc##1{\textcolor[rgb]{0.00,0.00,1.00}{##1}}}
\@namedef{PY@tok@vc}{\def\PY@tc##1{\textcolor[rgb]{0.10,0.09,0.49}{##1}}}
\@namedef{PY@tok@vg}{\def\PY@tc##1{\textcolor[rgb]{0.10,0.09,0.49}{##1}}}
\@namedef{PY@tok@vi}{\def\PY@tc##1{\textcolor[rgb]{0.10,0.09,0.49}{##1}}}
\@namedef{PY@tok@vm}{\def\PY@tc##1{\textcolor[rgb]{0.10,0.09,0.49}{##1}}}
\@namedef{PY@tok@sa}{\def\PY@tc##1{\textcolor[rgb]{0.73,0.13,0.13}{##1}}}
\@namedef{PY@tok@sb}{\def\PY@tc##1{\textcolor[rgb]{0.73,0.13,0.13}{##1}}}
\@namedef{PY@tok@sc}{\def\PY@tc##1{\textcolor[rgb]{0.73,0.13,0.13}{##1}}}
\@namedef{PY@tok@dl}{\def\PY@tc##1{\textcolor[rgb]{0.73,0.13,0.13}{##1}}}
\@namedef{PY@tok@s2}{\def\PY@tc##1{\textcolor[rgb]{0.73,0.13,0.13}{##1}}}
\@namedef{PY@tok@sh}{\def\PY@tc##1{\textcolor[rgb]{0.73,0.13,0.13}{##1}}}
\@namedef{PY@tok@s1}{\def\PY@tc##1{\textcolor[rgb]{0.73,0.13,0.13}{##1}}}
\@namedef{PY@tok@mb}{\def\PY@tc##1{\textcolor[rgb]{0.40,0.40,0.40}{##1}}}
\@namedef{PY@tok@mf}{\def\PY@tc##1{\textcolor[rgb]{0.40,0.40,0.40}{##1}}}
\@namedef{PY@tok@mh}{\def\PY@tc##1{\textcolor[rgb]{0.40,0.40,0.40}{##1}}}
\@namedef{PY@tok@mi}{\def\PY@tc##1{\textcolor[rgb]{0.40,0.40,0.40}{##1}}}
\@namedef{PY@tok@il}{\def\PY@tc##1{\textcolor[rgb]{0.40,0.40,0.40}{##1}}}
\@namedef{PY@tok@mo}{\def\PY@tc##1{\textcolor[rgb]{0.40,0.40,0.40}{##1}}}
\@namedef{PY@tok@ch}{\let\PY@it=\textit\def\PY@tc##1{\textcolor[rgb]{0.25,0.50,0.50}{##1}}}
\@namedef{PY@tok@cm}{\let\PY@it=\textit\def\PY@tc##1{\textcolor[rgb]{0.25,0.50,0.50}{##1}}}
\@namedef{PY@tok@cpf}{\let\PY@it=\textit\def\PY@tc##1{\textcolor[rgb]{0.25,0.50,0.50}{##1}}}
\@namedef{PY@tok@c1}{\let\PY@it=\textit\def\PY@tc##1{\textcolor[rgb]{0.25,0.50,0.50}{##1}}}
\@namedef{PY@tok@cs}{\let\PY@it=\textit\def\PY@tc##1{\textcolor[rgb]{0.25,0.50,0.50}{##1}}}

\def\PYZbs{\char`\\}
\def\PYZus{\char`\_}
\def\PYZob{\char`\{}
\def\PYZcb{\char`\}}
\def\PYZca{\char`\^}
\def\PYZam{\char`\&}
\def\PYZlt{\char`\<}
\def\PYZgt{\char`\>}
\def\PYZsh{\char`\#}
\def\PYZpc{\char`\%}
\def\PYZdl{\char`\$}
\def\PYZhy{\char`\-}
\def\PYZsq{\char`\'}
\def\PYZdq{\char`\"}
\def\PYZti{\char`\~}
% for compatibility with earlier versions
\def\PYZat{@}
\def\PYZlb{[}
\def\PYZrb{]}
\makeatother


    % For linebreaks inside Verbatim environment from package fancyvrb. 
    \makeatletter
        \newbox\Wrappedcontinuationbox 
        \newbox\Wrappedvisiblespacebox 
        \newcommand*\Wrappedvisiblespace {\textcolor{red}{\textvisiblespace}} 
        \newcommand*\Wrappedcontinuationsymbol {\textcolor{red}{\llap{\tiny$\m@th\hookrightarrow$}}} 
        \newcommand*\Wrappedcontinuationindent {3ex } 
        \newcommand*\Wrappedafterbreak {\kern\Wrappedcontinuationindent\copy\Wrappedcontinuationbox} 
        % Take advantage of the already applied Pygments mark-up to insert 
        % potential linebreaks for TeX processing. 
        %        {, <, #, %, $, ' and ": go to next line. 
        %        _, }, ^, &, >, - and ~: stay at end of broken line. 
        % Use of \textquotesingle for straight quote. 
        \newcommand*\Wrappedbreaksatspecials {% 
            \def\PYGZus{\discretionary{\char`\_}{\Wrappedafterbreak}{\char`\_}}% 
            \def\PYGZob{\discretionary{}{\Wrappedafterbreak\char`\{}{\char`\{}}% 
            \def\PYGZcb{\discretionary{\char`\}}{\Wrappedafterbreak}{\char`\}}}% 
            \def\PYGZca{\discretionary{\char`\^}{\Wrappedafterbreak}{\char`\^}}% 
            \def\PYGZam{\discretionary{\char`\&}{\Wrappedafterbreak}{\char`\&}}% 
            \def\PYGZlt{\discretionary{}{\Wrappedafterbreak\char`\<}{\char`\<}}% 
            \def\PYGZgt{\discretionary{\char`\>}{\Wrappedafterbreak}{\char`\>}}% 
            \def\PYGZsh{\discretionary{}{\Wrappedafterbreak\char`\#}{\char`\#}}% 
            \def\PYGZpc{\discretionary{}{\Wrappedafterbreak\char`\%}{\char`\%}}% 
            \def\PYGZdl{\discretionary{}{\Wrappedafterbreak\char`\$}{\char`\$}}% 
            \def\PYGZhy{\discretionary{\char`\-}{\Wrappedafterbreak}{\char`\-}}% 
            \def\PYGZsq{\discretionary{}{\Wrappedafterbreak\textquotesingle}{\textquotesingle}}% 
            \def\PYGZdq{\discretionary{}{\Wrappedafterbreak\char`\"}{\char`\"}}% 
            \def\PYGZti{\discretionary{\char`\~}{\Wrappedafterbreak}{\char`\~}}% 
        } 
        % Some characters . , ; ? ! / are not pygmentized. 
        % This macro makes them "active" and they will insert potential linebreaks 
        \newcommand*\Wrappedbreaksatpunct {% 
            \lccode`\~`\.\lowercase{\def~}{\discretionary{\hbox{\char`\.}}{\Wrappedafterbreak}{\hbox{\char`\.}}}% 
            \lccode`\~`\,\lowercase{\def~}{\discretionary{\hbox{\char`\,}}{\Wrappedafterbreak}{\hbox{\char`\,}}}% 
            \lccode`\~`\;\lowercase{\def~}{\discretionary{\hbox{\char`\;}}{\Wrappedafterbreak}{\hbox{\char`\;}}}% 
            \lccode`\~`\:\lowercase{\def~}{\discretionary{\hbox{\char`\:}}{\Wrappedafterbreak}{\hbox{\char`\:}}}% 
            \lccode`\~`\?\lowercase{\def~}{\discretionary{\hbox{\char`\?}}{\Wrappedafterbreak}{\hbox{\char`\?}}}% 
            \lccode`\~`\!\lowercase{\def~}{\discretionary{\hbox{\char`\!}}{\Wrappedafterbreak}{\hbox{\char`\!}}}% 
            \lccode`\~`\/\lowercase{\def~}{\discretionary{\hbox{\char`\/}}{\Wrappedafterbreak}{\hbox{\char`\/}}}% 
            \catcode`\.\active
            \catcode`\,\active 
            \catcode`\;\active
            \catcode`\:\active
            \catcode`\?\active
            \catcode`\!\active
            \catcode`\/\active 
            \lccode`\~`\~ 	
        }
    \makeatother

    \let\OriginalVerbatim=\Verbatim
    \makeatletter
    \renewcommand{\Verbatim}[1][1]{%
        %\parskip\z@skip
        \sbox\Wrappedcontinuationbox {\Wrappedcontinuationsymbol}%
        \sbox\Wrappedvisiblespacebox {\FV@SetupFont\Wrappedvisiblespace}%
        \def\FancyVerbFormatLine ##1{\hsize\linewidth
            \vtop{\raggedright\hyphenpenalty\z@\exhyphenpenalty\z@
                \doublehyphendemerits\z@\finalhyphendemerits\z@
                \strut ##1\strut}%
        }%
        % If the linebreak is at a space, the latter will be displayed as visible
        % space at end of first line, and a continuation symbol starts next line.
        % Stretch/shrink are however usually zero for typewriter font.
        \def\FV@Space {%
            \nobreak\hskip\z@ plus\fontdimen3\font minus\fontdimen4\font
            \discretionary{\copy\Wrappedvisiblespacebox}{\Wrappedafterbreak}
            {\kern\fontdimen2\font}%
        }%
        
        % Allow breaks at special characters using \PYG... macros.
        \Wrappedbreaksatspecials
        % Breaks at punctuation characters . , ; ? ! and / need catcode=\active 	
        \OriginalVerbatim[#1,codes*=\Wrappedbreaksatpunct]%
    }
    \makeatother

    % Exact colors from NB
    \definecolor{incolor}{HTML}{303F9F}
    \definecolor{outcolor}{HTML}{D84315}
    \definecolor{cellborder}{HTML}{CFCFCF}
    \definecolor{cellbackground}{HTML}{F7F7F7}
    
    % prompt
    \makeatletter
    \newcommand{\boxspacing}{\kern\kvtcb@left@rule\kern\kvtcb@boxsep}
    \makeatother
    \newcommand{\prompt}[4]{
        {\ttfamily\llap{{\color{#2}[#3]:\hspace{3pt}#4}}\vspace{-\baselineskip}}
    }
    

    
    % Prevent overflowing lines due to hard-to-break entities
    \sloppy 
    % Setup hyperref package
    \hypersetup{
      breaklinks=true,  % so long urls are correctly broken across lines
      colorlinks=true,
      urlcolor=urlcolor,
      linkcolor=linkcolor,
      citecolor=citecolor,
      }
    % Slightly bigger margins than the latex defaults
    
    \geometry{verbose,tmargin=1in,bmargin=1in,lmargin=1in,rmargin=1in}
    
    

\begin{document}
    
    \maketitle
    
    

    
    \emph{Sjekk først at du har installert pakkene riktig ved å kjøre
kodeblokken under:}

    \begin{tcolorbox}[breakable, size=fbox, boxrule=1pt, pad at break*=1mm,colback=cellbackground, colframe=cellborder]
\prompt{In}{incolor}{2}{\boxspacing}
\begin{Verbatim}[commandchars=\\\{\}]
\PY{k+kn}{from} \PY{n+nn}{graph\PYZus{}utils}\PY{n+nn}{.}\PY{n+nn}{graph} \PY{k+kn}{import} \PY{n}{Graph}\PY{p}{,} \PY{n}{nx}\PY{p}{,} \PY{n}{plt}
\PY{k+kn}{from} \PY{n+nn}{graph\PYZus{}utils}\PY{n+nn}{.}\PY{n+nn}{buss\PYZus{}graph} \PY{k+kn}{import} \PY{n}{BussGraph}
\PY{k+kn}{from} \PY{n+nn}{graph\PYZus{}utils}\PY{n+nn}{.}\PY{n+nn}{tree\PYZus{}graph} \PY{k+kn}{import} \PY{n}{TreeGraph}
\PY{k+kn}{from} \PY{n+nn}{graph\PYZus{}utils}\PY{n+nn}{.}\PY{n+nn}{ring\PYZus{}graph} \PY{k+kn}{import} \PY{n}{RingGraph}
\PY{k+kn}{from} \PY{n+nn}{graph\PYZus{}utils}\PY{n+nn}{.}\PY{n+nn}{star\PYZus{}graph} \PY{k+kn}{import} \PY{n}{StarGraph}
\PY{k+kn}{from} \PY{n+nn}{graph\PYZus{}utils}\PY{n+nn}{.}\PY{n+nn}{grid\PYZus{}graph} \PY{k+kn}{import} \PY{n}{GridGraph}
\PY{k+kn}{from} \PY{n+nn}{graph\PYZus{}utils}\PY{n+nn}{.}\PY{n+nn}{mesh\PYZus{}graph} \PY{k+kn}{import} \PY{n}{MeshGraph}
\PY{k+kn}{from} \PY{n+nn}{graph\PYZus{}utils}\PY{n+nn}{.}\PY{n+nn}{constructed\PYZus{}graph} \PY{k+kn}{import} \PY{n}{ConstructedGraph}
\PY{k+kn}{from} \PY{n+nn}{graph\PYZus{}utils}\PY{n+nn}{.}\PY{n+nn}{watts\PYZus{}strogatz} \PY{k+kn}{import} \PY{n}{WattsStrogatz}
\PY{k+kn}{from} \PY{n+nn}{graph\PYZus{}utils}\PY{n+nn}{.}\PY{n+nn}{barabasi\PYZus{}albert} \PY{k+kn}{import} \PY{n}{BarabasiAlbert}
\PY{k+kn}{from} \PY{n+nn}{graph\PYZus{}utils}\PY{n+nn}{.}\PY{n+nn}{real\PYZus{}network\PYZus{}graph} \PY{k+kn}{import} \PY{n}{RealNetworkGraph}
\PY{k+kn}{from} \PY{n+nn}{graph\PYZus{}utils}\PY{n+nn}{.}\PY{n+nn}{vdes\PYZus{}graph} \PY{k+kn}{import} \PY{n}{VDESGraph}
\PY{k+kn}{from} \PY{n+nn}{graph\PYZus{}utils}\PY{n+nn}{.}\PY{n+nn}{get\PYZus{}network} \PY{k+kn}{import} \PY{n}{get\PYZus{}network}
\PY{k+kn}{import} \PY{n+nn}{matplotlib}
\PY{k+kn}{import} \PY{n+nn}{numpy} \PY{k}{as} \PY{n+nn}{np}
\PY{k+kn}{import} \PY{n+nn}{random} \PY{k}{as} \PY{n+nn}{r}
\end{Verbatim}
\end{tcolorbox}

    \emph{Gå nøye gjennom Introduksjonsnotebooken før du gjør øvingen. Alle
verktøy du vil trenge presenteres der.}

    \begin{tcolorbox}[breakable, size=fbox, boxrule=1pt, pad at break*=1mm,colback=cellbackground, colframe=cellborder]
\prompt{In}{incolor}{3}{\boxspacing}
\begin{Verbatim}[commandchars=\\\{\}]
\PY{n}{student\PYZus{}seed} \PY{o}{=} \PY{l+m+mi}{10027} \PY{c+c1}{\PYZsh{}ALso student with student\PYZus{}seed: 10064}
\end{Verbatim}
\end{tcolorbox}

    \hypertarget{introduksjon}{%
\section*{Introduksjon}\label{introduksjon}}

I denne øvingen skal vi lære om ulike måter å måle viktigheten, eller
sentraliteten, til en node i et nettverk. Vi skal også se på ulike
graftyper, lære enkel analyse og til slutt se på tilfeldige og
målrettede feil i et nettverk.

Følgende temaer er forventet at du skal kunne etter denne øvingen: *
Sentraliteter * Mål på sammenkobling av nettverk * Noder i største
partisjon * Angrepsmål i et nettverk * Viktigheten av en node *
Angrepsstrategier i nettverk * Robusthet/utholdenhet * Redundans *
Kostnad ved redundans/trade-offs og effekten av dette

For å gjennomføre denne øvingen forventes det at man har gjennomført
introduksjonen, og kan bruke verktøyene som ble introdusert der til mer
krevende oppgaver. For oppgavene nedenfor forventes det at studenten
klarer å svare kort og konkret.

    \hypertarget{del-1-grafstrukturer-og-centralities}{%
\section*{Del 1: Grafstrukturer og
centralities}\label{del-1-grafstrukturer-og-centralities}}

I denne delen skal vi introdusere mål på viktigheten for en node i et
nettverk. Her skal vi lære om tre forskjellige standariserte mål: *
degree centrality * betweenness centrality * closeness centrality.

Vi skal også ta en nærmere titt på ulike graftyper og analysere deres
fordeler og ulemper.

    \hypertarget{oppgave-1.1}{%
\subsection*{Oppgave 1.1}\label{oppgave-1.1}}

Enkle grafstrukturer som er mye brukt er mesh, star, ring, tree og bus.
Ved hjelp av metodene introdusert i introduksjonen, konstruer og print
en enkel versjon av hver graf (5-10 noder).

Her forventer vi å se de fem graftypene konstruert og printet

    \begin{tcolorbox}[breakable, size=fbox, boxrule=1pt, pad at break*=1mm,colback=cellbackground, colframe=cellborder]
\prompt{In}{incolor}{9}{\boxspacing}
\begin{Verbatim}[commandchars=\\\{\}]
\PY{c+c1}{\PYZsh{}Bus}
\PY{n}{BussGraph}\PY{p}{(}\PY{l+m+mi}{5}\PY{p}{)}\PY{o}{.}\PY{n}{draw}\PY{p}{(}\PY{p}{)}
\end{Verbatim}
\end{tcolorbox}

    \begin{center}
    \adjustimage{max size={0.9\linewidth}{0.9\paperheight}}{Oving3-revidert-v1_files/Oving3-revidert-v1_7_0.png}
    \end{center}
    { \hspace*{\fill} \\}
    
    \begin{tcolorbox}[breakable, size=fbox, boxrule=1pt, pad at break*=1mm,colback=cellbackground, colframe=cellborder]
\prompt{In}{incolor}{5}{\boxspacing}
\begin{Verbatim}[commandchars=\\\{\}]
\PY{c+c1}{\PYZsh{}Ring}
\PY{n}{RingGraph}\PY{p}{(}\PY{l+m+mi}{5}\PY{p}{)}\PY{o}{.}\PY{n}{draw}\PY{p}{(}\PY{p}{)}
\end{Verbatim}
\end{tcolorbox}

    \begin{center}
    \adjustimage{max size={0.9\linewidth}{0.9\paperheight}}{Oving3-revidert-v1_files/Oving3-revidert-v1_8_0.png}
    \end{center}
    { \hspace*{\fill} \\}
    
    \begin{tcolorbox}[breakable, size=fbox, boxrule=1pt, pad at break*=1mm,colback=cellbackground, colframe=cellborder]
\prompt{In}{incolor}{6}{\boxspacing}
\begin{Verbatim}[commandchars=\\\{\}]
\PY{c+c1}{\PYZsh{}Star}
\PY{n}{StarGraph}\PY{p}{(}\PY{l+m+mi}{5}\PY{p}{)}\PY{o}{.}\PY{n}{draw}\PY{p}{(}\PY{p}{)}
\end{Verbatim}
\end{tcolorbox}

    \begin{center}
    \adjustimage{max size={0.9\linewidth}{0.9\paperheight}}{Oving3-revidert-v1_files/Oving3-revidert-v1_9_0.png}
    \end{center}
    { \hspace*{\fill} \\}
    
    \begin{tcolorbox}[breakable, size=fbox, boxrule=1pt, pad at break*=1mm,colback=cellbackground, colframe=cellborder]
\prompt{In}{incolor}{6}{\boxspacing}
\begin{Verbatim}[commandchars=\\\{\}]
\PY{c+c1}{\PYZsh{}Tree}
\PY{n}{TreeGraph}\PY{p}{(}\PY{l+m+mi}{2}\PY{p}{,}\PY{l+m+mi}{2}\PY{p}{)}\PY{o}{.}\PY{n}{draw}\PY{p}{(}\PY{p}{)}
\end{Verbatim}
\end{tcolorbox}

    \begin{center}
    \adjustimage{max size={0.9\linewidth}{0.9\paperheight}}{Oving3-revidert-v1_files/Oving3-revidert-v1_10_0.png}
    \end{center}
    { \hspace*{\fill} \\}
    
    \begin{tcolorbox}[breakable, size=fbox, boxrule=1pt, pad at break*=1mm,colback=cellbackground, colframe=cellborder]
\prompt{In}{incolor}{7}{\boxspacing}
\begin{Verbatim}[commandchars=\\\{\}]
\PY{c+c1}{\PYZsh{}Mesh}
\PY{n}{MeshGraph}\PY{p}{(}\PY{l+m+mi}{5}\PY{p}{)}\PY{o}{.}\PY{n}{draw}\PY{p}{(}\PY{p}{)}
\end{Verbatim}
\end{tcolorbox}

    \begin{center}
    \adjustimage{max size={0.9\linewidth}{0.9\paperheight}}{Oving3-revidert-v1_files/Oving3-revidert-v1_11_0.png}
    \end{center}
    { \hspace*{\fill} \\}
    
    \hypertarget{oppgave-1.2}{%
\subsection*{Oppgave 1.2}\label{oppgave-1.2}}

For hver av strukturene gitt i oppgave 1.1, kom med et eksempel fra
virkeligheten som bruker denne strukturen. For hver av strukturene,
kommenter hva som er styrker og svakheter.

For hver graf forventer vi å se minst ett eksempel på hvor denne
strukturen brukes. Vi forventer også en kort drøftning på styrker og
svakheter

    \hypertarget{skriv-svar-her}{%
\section*{Skriv svar her:}\label{skriv-svar-her}}

\hypertarget{bussgraph}{%
\subsection*{BussGraph}\label{bussgraph}}

I en et lineært BussGraph-nettverk er alle eneheten koblet til samme
ryggrad eller buss, bestående av en nettverks-kabel eller lignenende.

\hypertarget{eksempel-ethernet}{%
\subsubsection*{Eksempel: Ethernet}\label{eksempel-ethernet}}

Det mest relevante eksempelet idag vil være et tradisjonel ethernet
nettverk, som kan betegnes som et ``Local Area Network''(LAN). I et
slikt nettverk vil enhver pakke bli sendt til alle enhetene i
nettverket, med en MAC-adresse til mottaker. Dermed vil bare
datamaskinen med den gitte MAC-Adressen behandle pakken.

\hypertarget{styrker}{%
\subsubsection*{Styrker:}\label{styrker}}

\begin{itemize}
\tightlist
\item
  Krever mindre ressurser med tanke på opsett, altså færre kabler.
\item
  Enkelt å sette opp og konfigurere
\item
  En feil i en enhet reduserer ikke ytelsen av resten av nettverket,
  annet enn dens egen funskjon.
\end{itemize}

\hypertarget{svakheter}{%
\subsubsection*{Svakheter}\label{svakheter}}

\begin{itemize}
\tightlist
\item
  Om ryggraden av nettverket feiler, kollapser hele nettverket.
\item
  Ekstremt sakte i forhold til andre typer nettverk, ikke brukbart i
  stor skala.
\item
  Ytelsen reduseres om en legger til flere enheter i nettverket.
\end{itemize}

\hypertarget{ringgraph}{%
\subsection*{RingGraph}\label{ringgraph}}

Ring graf viser en nettverk av noder som alle har nøyaktig 2 kanter, og
man kan alltid ``flytte'' på nodene slik at det ser ut som en ring, som
kalles ring topologi.

\hypertarget{eksempel-token-passing}{%
\subsubsection*{Eksempel: Token Passing}\label{eksempel-token-passing}}

En type MAC-protokoll som ble brukt i datanettverk før var ``Token
passing''. Dette protokollet eliminerer kollisjoner i
kommunikasjon-nettverket mellom datmaskiner ved at en såkalt ``token''
ble gitt rundt, som tillot en gitt enheten å være den eneste med tilgang
til å sende informasjon over kanalen. Denne ``Token'' går på rundgang i
nettverket for å få en lik fordeling mellom alle de gitte enhetene.

\hypertarget{styrker-1}{%
\subsubsection*{Styrker}\label{styrker-1}}

\begin{itemize}
\tightlist
\item
  Om en node feiler er det lett å detektere at det er en feil fordi
  grafen ikke lenger vil ha en sykel
\item
  Jevn fordeling av kanter på nodene
\end{itemize}

\hypertarget{svakheter-1}{%
\subsubsection*{Svakheter}\label{svakheter-1}}

\begin{itemize}
\tightlist
\item
  Om man en node feiler kan det risikere å ødelegge hele nettverket om
  man ikke har gode protokoller for det.
\item
  I Token passing vil Token alltid blir sendt rundt til alle nodene,
  selv om en node ikke skal sende noe data, dette bidrar til unødvendig
  bruk av resurser.
\item
  Om man skal legge til / fjerne, kan det påvirke de andre
  ende-systemene fordi man er nødt til å finne en ny ring før
  \emph{Token}3 kan sendes videre
\end{itemize}

\hypertarget{stargraph}{%
\subsection*{StarGraph}\label{stargraph}}

En slikt nettverke er et utelukkende sentralt nettverk der en har én
node med en til alle relasjon som en kaller ``hub'', og resten av nodene
har én til én relasjon til huben.

\hypertarget{eksempel-lan-switch}{%
\subsubsection*{Eksempel: LAN Switch}\label{eksempel-lan-switch}}

I de flest LAN nettverk i dag, brukes en switch som kan sees på som en
hub, som kommuniserer med alle ende-systemene. At all trafikk går
gjennom en hub kan være veldig oversiktlig og elegant, men om huben
feiler, vil hele nettverket kollapse. Å sikre at huben både tåler
belastningen av nettverket og er robust, er sentralt for å kunne benytte
seg av en slik metode.

\hypertarget{styrker-2}{%
\subsubsection*{Styrker}\label{styrker-2}}

\begin{itemize}
\tightlist
\item
  Enkelt og billig å sette opp
\item
  Hvis et ende-system feiler, vil det ikke ha noe negativ påvirkning på
  de andre ende-systemene
\item
  Det går veldig raskt om det er lav datatrafikk, fordi man kan sende
  data med én gang inn til hubben uten å måtte vente på f.eks. en Token
\end{itemize}

\hypertarget{svakheter-2}{%
\subsubsection*{Svakheter}\label{svakheter-2}}

\begin{itemize}
\tightlist
\item
  Kan være sårbart med et så sentralt nettverk; Feiler huben, kollapser
  hele nettet.
\item
  Høy trafikk kan redusere ytelsen av nettverket betydelig.
\item
  Kostbart å installere nettverket.
\end{itemize}

\hypertarget{treegraph}{%
\subsection*{TreeGraph}\label{treegraph}}

I et slikt nettverk har man en rotnode som blir ekspandert på av resten
av nettverket på en hierarkisk måte.

\hypertarget{eksempel}{%
\subsubsection*{Eksempel :}\label{eksempel}}

\hypertarget{styrker-3}{%
\subsubsection*{Styrker}\label{styrker-3}}

\begin{itemize}
\tightlist
\item
  Lettvint å ekspandere nettverket
\item
  Feil-håndtering er også lettvint
\end{itemize}

\hypertarget{svakheter-3}{%
\subsubsection*{Svakheter}\label{svakheter-3}}

\begin{itemize}
\tightlist
\item
  Hvis rotnoden feiler, kollapser hele nettverket
\item
  Vanskelig å konfigurere
\end{itemize}

\hypertarget{meshgraph}{%
\subsection*{MeshGraph}\label{meshgraph}}

I et Mesh-nettverk er korteste vei mellom en til alle enehtene 1 kant,
og en har mange muligheter for å nå samme punkt. Det gjør dermed Mesh
til en svært robust måte å bygge nettverk på.

\hypertarget{eksempel-peer-to-peer}{%
\subsubsection*{Eksempel Peer-To-Peer:}\label{eksempel-peer-to-peer}}

P2P-nettverk har en desentralisert struktur der enhver enhet er
``likverdige''. I et slikt nettverk deler man ressurser og igjen arbeid
mot nettverkets ``formål''. I et slikt nettverk er ofte hver node både
bruker og leverandør, som igjen gjør det svært robust.

\hypertarget{styrker-4}{%
\subsubsection*{Styrker}\label{styrker-4}}

\begin{itemize}
\tightlist
\item
  Svært pålitleig kommunikasjon over nettverket. Om en kant eller node
  feiler vil det det være en alternativ vei å lever pakken.
\item
  Vanskelig å overbelaste nettverket da det er mange koblinger.
\end{itemize}

\hypertarget{svakheter-4}{%
\subsubsection*{Svakheter}\label{svakheter-4}}

\begin{itemize}
\tightlist
\item
  Kostbart å bygge nettverket, samt at det er vanskelig å konfigurere.
\item
  Kostbart å legge til flere enheter i nettverket.
\item
  Krever svært mye fysisk infrastruktur om en gjør det lokalt. Krav om
  mye kabler, porter og ikke minst plass.
\end{itemize}

    \hypertarget{oppgave-1.3}{%
\subsection*{Oppgave 1.3}\label{oppgave-1.3}}

For hver graf, finn noden med høyest degree centrality for hånd. Vis
utregning og forklar viktigheten en node med høy degree centrality
tilføyer nettverket.

Her forventer vi å se at man har forstått seg på hva begrepet degree
centrality handler om, og funnet den viktigste noden. Vi forventer også
å se utregning og at studenten ikke har funnet svaret med å bruke en
metode. Til slutt forventer vi en kort kommentar på hvorfor denne noden
er viktig.

    \hypertarget{bussgraph}{%
\subsection*{BussGraph}\label{bussgraph}}

\[\text{Formelen for degree centrality for hver node er:}\]

\[\frac{(\text{antall kanter noden har})}{(\text{antal noder i grafen - 1})}\]

\[\text{Node}_0 = \frac{1}{5 - 1} \implies \frac{1}{4} \implies \underline{\underline{ 0.25}}\]

\[\text{Node}_1 = \frac{2}{5 - 1} \implies \frac{2}{4} \implies \underline{\underline{ 0.5}}\]

\[\text{Node}_2 = \frac{2}{5 - 1} \implies \frac{2}{4} \implies \underline{\underline{ 0.5}}\]

\[\text{Node}_3 = \frac{2}{5 - 1} \implies \frac{2}{4} \implies \underline{\underline{ 0.5}}\]

\[\text{Node}_4 = \frac{1}{5 - 1} \implies \frac{1}{4} \implies \underline{\underline{ 0.25}}\]

Her ser vi at node 1,2 og 3 har høyest degree centrality på 0.5 i denne
grafen er det vitig at disse nodene ikke feiler, fordi om de feiler
deler vi nettverket i 2 partisjoner (regionalnett).

    \hypertarget{ringgraph}{%
\subsection*{RingGraph}\label{ringgraph}}

\[\text{Formelen for degree centrality for hver node er:}\]

\[\frac{(\text{antall kanter noden har})}{(\text{antal noder i grafen - 1})}\]

\[\text{Node}_0 = \frac{2}{5 - 1} \implies \frac{2}{4} \implies \underline{\underline{ 0.5}}\]

\[\text{Node}_1 = \frac{2}{5 - 1} \implies \frac{2}{4} \implies \underline{\underline{ 0.5}}\]

\[\text{Node}_2 = \frac{2}{5 - 1} \implies \frac{2}{4} \implies \underline{\underline{ 0.5}}\]

\[\text{Node}_3 = \frac{2}{5 - 1} \implies \frac{2}{4} \implies \underline{\underline{ 0.5}}\]

\[\text{Node}_4 = \frac{2}{5 - 1} \implies \frac{2}{4} \implies \underline{\underline{ 0.5}}\]

Her ser vi at alle nodene har like stor degree centrality på 0.5 fordi i
en ringgraf vi alle nodene ha to kanter.

    \hypertarget{stargraph}{%
\subsection*{StarGraph}\label{stargraph}}

\[\text{Formelen for degree centrality for hver node er:}\]

\[\frac{(\text{antall kanter noden har})}{(\text{antal noder i grafen - 1})}\]

\[\text{Node}_0 = \frac{5}{6 - 1} \implies \frac{5}{5} \implies \underline{\underline{ 1.0}}\]

\[\text{Node}_1 = \frac{1}{6 - 1} \implies \frac{1}{5} \implies \underline{\underline{ 0.2}}\]

\[\text{Node}_2 = \frac{1}{6 - 1} \implies \frac{1}{5} \implies \underline{\underline{ 0.2}}\]

\[\text{Node}_3 = \frac{1}{6 - 1} \implies \frac{1}{5} \implies \underline{\underline{ 0.2}}\]

\[\text{Node}_4 = \frac{1}{6 - 1} \implies \frac{1}{5} \implies \underline{\underline{ 0.2}}\]

\[\text{Node}_5 = \frac{1}{6 - 1} \implies \frac{1}{5} \implies \underline{\underline{ 0.2}}\]

Her ser vi at det er én node med vesentrlig høyere degree centraliy på
1, som er den høyeste verdien man kan ha, som vil se at den har alle
kantene i grafen. om denne node feiler vil alle kantene ``forsvinne'' og
ingen node i nettverket vil kunne kommunisere med hverandre.

    \hypertarget{treegraph}{%
\subsection*{TreeGraph:}\label{treegraph}}

\[\text{Formelen for degree centrality for hver node er:}\]

\[\frac{(\text{antall kanter noden har})}{(\text{antal noder i grafen - 1})}\]

\[\text{Node}_0 = \frac{2}{7 - 1} \implies \frac{2}{6} \implies \underline{\underline{ 0.33}}\]

\[\text{Node}_1 = \frac{3}{7 - 1} \implies \frac{3}{6} \implies \underline{\underline{ 0.5}}\]

\[\text{Node}_2 = \frac{3}{7 - 1} \implies \frac{3}{6} \implies \underline{\underline{ 0.5}}\]

\[\text{Node}_3 = \frac{1}{7 - 1} \implies \frac{1}{6} \implies \underline{\underline{ 0.17}}\]

\[\text{Node}_4 = \frac{1}{7 - 1} \implies \frac{1}{6} \implies \underline{\underline{ 0.17}}\]

\[\text{Node}_5 = \frac{1}{7 - 1} \implies \frac{1}{6} \implies \underline{\underline{ 0.17}}\]

\[\text{Node}_6 = \frac{1}{7 - 1} \implies \frac{1}{6} \implies \underline{\underline{ 0.17}}\]

Her ser vi at det er 2 noder med høyes degree centrality på 0.5, disse
nodene i dette tillfellet er essensielle fordi, de kobler alle nodene
sammen og uten disse to vil det ikke være noen kanter igjen i
nettverket.

    \hypertarget{meshgraph}{%
\subsection*{MeshGraph}\label{meshgraph}}

\[\text{Formelen for degree centrality for hver node er:}\]

\[\frac{(\text{antall kanter noden har})}{(\text{antal noder i grafen - 1})}\]

\[\text{Node}_0 = \frac{4}{5 - 1} \implies \frac{4}{4} \implies \underline{\underline{ 1.0}}\]

\[\text{Node}_1 = \frac{4}{5 - 1} \implies \frac{4}{4} \implies \underline{\underline{ 1.0}}\]

\[\text{Node}_2 = \frac{4}{5 - 1} \implies \frac{4}{4} \implies \underline{\underline{ 1.0}}\]

\[\text{Node}_3 = \frac{4}{5 - 1} \implies \frac{4}{4} \implies \underline{\underline{ 1.0}}\]

\[\text{Node}_4 = \frac{4}{5 - 1} \implies \frac{4}{4} \implies \underline{\underline{ 1.0}}\]

Her har alle nodene en maks-score på 1 i degree centrality, som vil si
at alle nodene i grafen har en direkte til hverandre. viktigheten til én
av disse nodene forsvinner litt da, fordi uansett hvilken node man
fjerner, vil fortsatt alle nodene har degree centrality på 1, og
fortsatt kunne kommunisere med direkte med alle de andre nodene.

    \hypertarget{oppgave-1.4}{%
\subsection*{Oppgave 1.4}\label{oppgave-1.4}}

For hver graf, finn noden med høyest betweenness centrality for hånd.
Vis utregning og forklar viktigheten en node med høy degree centrality
tilføyer nettverket.

Her forventer vi å se at man har forstått seg på hva begrepet
betweenness centrality handler om, og funnet den viktigste noden. Vi
forventer også å se utregning og at studenten ikke har funnet svaret med
å bruke en metode. Til slutt forventer vi en kort kommentar på hvorfor
denne noden er viktig

    \hypertarget{betweenness-centrality-for-bussgraph}{%
\subsection*{Betweenness centrality for
BussGraph}\label{betweenness-centrality-for-bussgraph}}

\[\text{Formelen betweenness centrality for en node er:}\]

\[\frac{\text{Antall ganger noden er i en korteste sti mellom alle nodepar}}{\frac{\text{(antall noder - 1)(antall noder - 2)}}{2}}\]

\[\text{Node}_0:\] \[\text{Korteste veiene noden er i:}\]

\[ [] \implies 0 \]

\[ \frac{0}{\frac{(5-1)*(5-2)}{2}} \implies \frac{0*2}{12} \implies \underline{\underline{ 0.0 }}\]

\[\text{Node}_1:\] \[\text{Korteste veiene noden er i:}\]

\[ [(0, 2), (0, 3), (0, 4)] \implies 3 \]

\[ \frac{3}{\frac{(5-1)*(5-2)}{2}} \implies \frac{3*2}{12} \implies \underline{\underline{ 0.5 }}\]

\[\text{Node}_2:\] \[\text{Korteste veiene noden er i:}\]

\[ [(0, 3), (0, 4), (1, 3), (1, 4)] \implies 4 \]

\[ \frac{4}{\frac{(5-1)*(5-2)}{2}} \implies \frac{4*2}{12} \implies \underline{\underline{ 0.67 }}\]

\[\text{Node}_3:\] \[\text{Korteste veiene noden er i:}\]

\[ [(0, 4), (1, 4), (2, 4)] \implies 3 \]

\[ \frac{3}{\frac{(5-1)*(5-2)}{2}} \implies \frac{3*2}{12} \implies \underline{\underline{ 0.5 }}\]

\[\text{Node}_4:\] \[\text{Korteste veiene noden er i:}\]

\[ [] \implies 0 \]

\[ \frac{0}{\frac{(5-1)*(5-2)}{2}} \implies \frac{0*2}{12} \implies \underline{\underline{ 0.0 }}\]

Her ser vi at det er én node som har høyest betweenness centrality på
0.67. Dette er fordi det er noden i midten av bussen, og da gir det
mening at den noden er inkludert i flest stier. Viktigheten til den
noden er stor på grunn av nettopp dette.

    \hypertarget{betweenness-centrality-for-ringgraph}{%
\subsection*{Betweenness centrality for
RingGraph}\label{betweenness-centrality-for-ringgraph}}

\[\text{Formelen betweenness centrality for en node er:}\]

\[\frac{\text{Antall ganger noden er i en korteste sti mellom alle nodepar}}{\frac{\text{(antall noder - 1)(antall noder - 2)}}{2}}\]

\[\text{Node}_0:\] \[\text{Korteste veiene noden er i:}\]

\[ [(1, 4)] \implies 1 \]

\[ \frac{1}{\frac{(5-1)*(5-2)}{2}} \implies \frac{1*2}{12} \implies \underline{\underline{ 0.17 }}\]

\[\text{Node}_1:\] \[\text{Korteste veiene noden er i:}\]

\[ [(0, 2)] \implies 1 \]

\[ \frac{1}{\frac{(5-1)*(5-2)}{2}} \implies \frac{1*2}{12} \implies \underline{\underline{ 0.17 }}\]

\[\text{Node}_2:\] \[\text{Korteste veiene noden er i:}\]

\[ [(1, 3)] \implies 1 \]

\[ \frac{1}{\frac{(5-1)*(5-2)}{2}} \implies \frac{1*2}{12} \implies \underline{\underline{ 0.17 }}\]

\[\text{Node}_3:\] \[\text{Korteste veiene noden er i:}\]

\[ [(2, 4)] \implies 1 \]

\[ \frac{1}{\frac{(5-1)*(5-2)}{2}} \implies \frac{1*2}{12} \implies \underline{\underline{ 0.17 }}\]

\[\text{Node}_4:\] \[\text{Korteste veiene noden er i:}\]

\[ [(0, 3)] \implies 1 \]

\[ \frac{1}{\frac{(5-1)*(5-2)}{2}} \implies \frac{1*2}{12} \implies \underline{\underline{ 0.17 }}\]

Her ser vi at alle nodene i ringgrafen har lik betweenness centrality på
0.17 bare, dette er et ganske lavt tall siden betweenness centrality gir
en standarisert score mellom 0 og 1.

    \hypertarget{betweenness-centrality-for-stargraph}{%
\subsection*{Betweenness centrality for
StarGraph}\label{betweenness-centrality-for-stargraph}}

\[\text{Formelen betweenness centrality for en node er:}\]

\[\frac{\text{Antall ganger noden er i en korteste sti mellom alle nodepar}}{\frac{\text{(antall noder - 1)(antall noder - 2)}}{2}}\]

\[\text{Node}_0:\] \[\text{Korteste veiene noden er i:}\]

\[ [(1, 2), (1, 3), (1, 4), (1, 5), (2, 3), (2, 4), (2, 5), (3, 4), (3, 5), (4, 5)] \implies 10 \]

\[ \frac{10}{\frac{(6-1)*(6-2)}{2}} \implies \frac{10*2}{20} \implies \underline{\underline{ 1.0 }}\]

\[\text{Node}_1:\] \[\text{Korteste veiene noden er i:}\]

\[ [] \implies 0 \]

\[ \frac{0}{\frac{(6-1)*(6-2)}{2}} \implies \frac{0*2}{20} \implies \underline{\underline{ 0.0 }}\]

\[\text{Node}_2:\] \[\text{Korteste veiene noden er i:}\]

\[ [] \implies 0 \]

\[ \frac{0}{\frac{(6-1)*(6-2)}{2}} \implies \frac{0*2}{20} \implies \underline{\underline{ 0.0 }}\]

\[\text{Node}_3:\] \[\text{Korteste veiene noden er i:}\]

\[ [] \implies 0 \]

\[ \frac{0}{\frac{(6-1)*(6-2)}{2}} \implies \frac{0*2}{20} \implies \underline{\underline{ 0.0 }}\]

\[\text{Node}_4:\] \[\text{Korteste veiene noden er i:}\]

\[ [] \implies 0 \]

\[ \frac{0}{\frac{(6-1)*(6-2)}{2}} \implies \frac{0*2}{20} \implies \underline{\underline{ 0.0 }}\]

\[\text{Node}_5:\] \[\text{Korteste veiene noden er i:}\]

\[ [] \implies 0 \]

\[ \frac{0}{\frac{(6-1)*(6-2)}{2}} \implies \frac{0*2}{20} \implies \underline{\underline{ 0.0 }}\]

Her ser vi at alle nodene har betweenness centrality på 0, unatt en node
som har betweenness centrality på 1, dette er en vedlig sentral node i
grafen. Siden alle andre noder har betweenness centrality på 0, vil det
si at node 0 er den eneste som kan være med i stien uten at den er
start- eller ende-noden.

    \hypertarget{betweenness-centrality-for-treegraph}{%
\subsection*{Betweenness centrality for
TreeGraph}\label{betweenness-centrality-for-treegraph}}

\[\text{Formelen betweenness centrality for en node er:}\]

\[\frac{\text{Antall ganger noden er i en korteste sti mellom alle nodepar}}{\frac{\text{(antall noder - 1)(antall noder - 2)}}{2}}\]

\[\text{Node}_0:\] \[\text{Korteste veiene noden er i:}\]

\[ [(1, 2), (1, 5), (1, 6), (2, 3), (2, 4), (3, 5), (3, 6), (4, 5), (4, 6)] \implies 9 \]

\[ \frac{9}{\frac{(7-1)*(7-2)}{2}} \implies \frac{9*2}{30} \implies \underline{\underline{ 0.6 }}\]

\[\text{Node}_1:\] \[\text{Korteste veiene noden er i:}\]

\[ [(0, 3), (0, 4), (2, 3), (2, 4), (3, 4), (3, 5), (3, 6), (4, 5), (4, 6)] \implies 9 \]

\[ \frac{9}{\frac{(7-1)*(7-2)}{2}} \implies \frac{9*2}{30} \implies \underline{\underline{ 0.6 }}\]

\[\text{Node}_2:\] \[\text{Korteste veiene noden er i:}\]

\[ [(0, 5), (0, 6), (1, 5), (1, 6), (3, 5), (3, 6), (4, 5), (4, 6), (5, 6)] \implies 9 \]

\[ \frac{9}{\frac{(7-1)*(7-2)}{2}} \implies \frac{9*2}{30} \implies \underline{\underline{ 0.6 }}\]

\[\text{Node}_3:\] \[\text{Korteste veiene noden er i:}\]

\[ [] \implies 0 \]

\[ \frac{0}{\frac{(7-1)*(7-2)}{2}} \implies \frac{0*2}{30} \implies \underline{\underline{ 0.0 }}\]

\[\text{Node}_4:\] \[\text{Korteste veiene noden er i:}\]

\[ [] \implies 0 \]

\[ \frac{0}{\frac{(7-1)*(7-2)}{2}} \implies \frac{0*2}{30} \implies \underline{\underline{ 0.0 }}\]

\[\text{Node}_5:\] \[\text{Korteste veiene noden er i:}\]

\[ [] \implies 0 \]

\[ \frac{0}{\frac{(7-1)*(7-2)}{2}} \implies \frac{0*2}{30} \implies \underline{\underline{ 0.0 }}\]

\[\text{Node}_6:\] \[\text{Korteste veiene noden er i:}\]

\[ [] \implies 0 \]

\[ \frac{0}{\frac{(7-1)*(7-2)}{2}} \implies \frac{0*2}{30} \implies \underline{\underline{ 0.0 }}\]

Her er det 3 noder med betweenness centrality på 0.6, mens resten av
nodene har betweenness centrality på 0. Da vil disse nodene være veldig
viktige når det kommer til stier mellom nodene. mens de på 0 vil ikke ha
noen betydning når det kommer til å danne stier videre.

    \hypertarget{betweenness-centrality-for-meshgraph}{%
\subsection*{Betweenness centrality for
MeshGraph}\label{betweenness-centrality-for-meshgraph}}

\[\text{Formelen betweenness centrality for en node er:}\]

\[\frac{\text{Antall ganger noden er i en korteste sti mellom alle nodepar}}{\frac{\text{(antall noder - 1)(antall noder - 2)}}{2}}\]

\[\text{Node}_0:\] \[\text{Korteste veiene noden er i:}\]

\[ [] \implies 0 \]

\[ \frac{0}{\frac{(5-1)*(5-2)}{2}} \implies \frac{0*2}{12} \implies \underline{\underline{ 0.0 }}\]

\[\text{Node}_1:\] \[\text{Korteste veiene noden er i:}\]

\[ [] \implies 0 \]

\[ \frac{0}{\frac{(5-1)*(5-2)}{2}} \implies \frac{0*2}{12} \implies \underline{\underline{ 0.0 }}\]

\[\text{Node}_2:\] \[\text{Korteste veiene noden er i:}\]

\[ [] \implies 0 \]

\[ \frac{0}{\frac{(5-1)*(5-2)}{2}} \implies \frac{0*2}{12} \implies \underline{\underline{ 0.0 }}\]

\[\text{Node}_3:\] \[\text{Korteste veiene noden er i:}\]

\[ [] \implies 0 \]

\[ \frac{0}{\frac{(5-1)*(5-2)}{2}} \implies \frac{0*2}{12} \implies \underline{\underline{ 0.0 }}\]

\[\text{Node}_4:\] \[\text{Korteste veiene noden er i:}\]

\[ [] \implies 0 \]

\[ \frac{0}{\frac{(5-1)*(5-2)}{2}} \implies \frac{0*2}{12} \implies \underline{\underline{ 0.0 }}\]

Her ser vi fort at alle nodene har betweenness centrality på 0. Dette
kommer av at alle nodene har en direkte sti til hverandre, og derfor
trenger man ikke å bruke andre noder til å finne korteste sti til en
annen node. Det vil også si at nodene ikke er viktige når det kommer til
å finne stier til andre noder.

    \hypertarget{oppgave-1.5}{%
\subsection*{Oppgave 1.5}\label{oppgave-1.5}}

For hver graf, finn noden med høyest closeness centrality for hånd. Vis
utregning og forklar viktigheten en node med høy degree centrality
tilføyer nettverket.

Her forventer vi å se at man har forstått seg på hva begrepet closeness
centrality handler om, og funnet den viktigste noden. Vi forventer også
å se utregning og at studenten ikke har funnet svaret med å bruke en
metode. Til slutt forventer vi en kort kommentar på hvorfor denne noden
er viktig

    \hypertarget{closesness-centrality-for-bussgraph}{%
\subsection*{Closesness centrality for
BussGraph:}\label{closesness-centrality-for-bussgraph}}

\[\text{Formelen for closesness centrality for en node er:}\]

\[\frac{\text{Antall noder - 1}}{\text{Den totale avstanden fra den noden til alle andre noder}}\]

\textbf{Node 0}, den korteste veien til hver node:

Korteste vei fra 0 til 1 = {[}1{]} = 1

Korteste vei fra 0 til 2 = {[}1, 2{]} = 2

Korteste vei fra 0 til 3 = {[}1, 2, 3{]} = 3

Korteste vei fra 0 til 4 = {[}1, 2, 3, 4{]} = 4

Den totale lengden: 10
\[\text{Closeness centrality} = \frac{5 - 1}{10} \implies \frac{4}{10} \implies\underline{\underline{ 0.4 }}\]

\textbf{Node 1}, den korteste veien til hver node:

Korteste vei fra 1 til 0 = {[}0{]} = 1

Korteste vei fra 1 til 2 = {[}2{]} = 1

Korteste vei fra 1 til 3 = {[}2, 3{]} = 2

Korteste vei fra 1 til 4 = {[}2, 3, 4{]} = 3

Den totale lengden: 7
\[\text{Closeness centrality} = \frac{5 - 1}{7} \implies \frac{4}{7} \implies\underline{\underline{ 0.57 }}\]

\textbf{Node 2}, den korteste veien til hver node:

Korteste vei fra 2 til 0 = {[}1, 0{]} = 2

Korteste vei fra 2 til 1 = {[}1{]} = 1

Korteste vei fra 2 til 3 = {[}3{]} = 1

Korteste vei fra 2 til 4 = {[}3, 4{]} = 2

Den totale lengden: 6
\[\text{Closeness centrality} = \frac{5 - 1}{6} \implies \frac{4}{6} \implies\underline{\underline{ 0.67 }}\]

\textbf{Node 3}, den korteste veien til hver node:

Korteste vei fra 3 til 0 = {[}2, 1, 0{]} = 3

Korteste vei fra 3 til 1 = {[}2, 1{]} = 2

Korteste vei fra 3 til 2 = {[}2{]} = 1

Korteste vei fra 3 til 4 = {[}4{]} = 1

Den totale lengden: 7
\[\text{Closeness centrality} = \frac{5 - 1}{7} \implies \frac{4}{7} \implies\underline{\underline{ 0.57 }}\]

\textbf{Node 4}, den korteste veien til hver node:

Korteste vei fra 4 til 0 = {[}3, 2, 1, 0{]} = 4

Korteste vei fra 4 til 1 = {[}3, 2, 1{]} = 3

Korteste vei fra 4 til 2 = {[}3, 2{]} = 2

Korteste vei fra 4 til 3 = {[}3{]} = 1

Den totale lengden: 10
\[\text{Closeness centrality} = \frac{5 - 1}{10} \implies \frac{4}{10} \implies\underline{\underline{ 0.4 }}\]

Noden med høyest closeness centrality er node 2, og dette kommer av at
den er i midten av bussgrafen og har da kortest totale vei. Det vil si
at den vil i snitt ha kortere vei til en tilfeldig node i grafen.

    \hypertarget{closesness-centrality-for-ringgraph}{%
\subsection*{Closesness centrality for
RingGraph:}\label{closesness-centrality-for-ringgraph}}

\[\text{Formelen for closesness centrality for en node er:}\]

\[\frac{\text{Antall noder - 1}}{\text{Den totale avstanden fra den noden til alle andre noder}}\]

\textbf{Node 0}, den korteste veien til hver node:

Korteste vei fra 0 til 1 = {[}1{]} = 1

Korteste vei fra 0 til 2 = {[}1, 2{]} = 2

Korteste vei fra 0 til 3 = {[}4, 3{]} = 2

Korteste vei fra 0 til 4 = {[}4{]} = 1

Den totale lengden: 6
\[\text{Closeness centrality} = \frac{5 - 1}{6} \implies \frac{4}{6} \implies\underline{\underline{ 0.67 }}\]

\textbf{Node 1}, den korteste veien til hver node:

Korteste vei fra 1 til 0 = {[}0{]} = 1

Korteste vei fra 1 til 2 = {[}2{]} = 1

Korteste vei fra 1 til 3 = {[}2, 3{]} = 2

Korteste vei fra 1 til 4 = {[}0, 4{]} = 2

Den totale lengden: 6
\[\text{Closeness centrality} = \frac{5 - 1}{6} \implies \frac{4}{6} \implies\underline{\underline{ 0.67 }}\]

\textbf{Node 2}, den korteste veien til hver node:

Korteste vei fra 2 til 0 = {[}1, 0{]} = 2

Korteste vei fra 2 til 1 = {[}1{]} = 1

Korteste vei fra 2 til 3 = {[}3{]} = 1

Korteste vei fra 2 til 4 = {[}3, 4{]} = 2

Den totale lengden: 6
\[\text{Closeness centrality} = \frac{5 - 1}{6} \implies \frac{4}{6} \implies\underline{\underline{ 0.67 }}\]

\textbf{Node 3}, den korteste veien til hver node:

Korteste vei fra 3 til 0 = {[}4, 0{]} = 2

Korteste vei fra 3 til 1 = {[}2, 1{]} = 2

Korteste vei fra 3 til 2 = {[}2{]} = 1

Korteste vei fra 3 til 4 = {[}4{]} = 1

Den totale lengden: 6
\[\text{Closeness centrality} = \frac{5 - 1}{6} \implies \frac{4}{6} \implies\underline{\underline{ 0.67 }}\]

\textbf{Node 4}, den korteste veien til hver node:

Korteste vei fra 4 til 0 = {[}0{]} = 1

Korteste vei fra 4 til 1 = {[}0, 1{]} = 2

Korteste vei fra 4 til 2 = {[}3, 2{]} = 2

Korteste vei fra 4 til 3 = {[}3{]} = 1

Den totale lengden: 6
\[\text{Closeness centrality} = \frac{5 - 1}{6} \implies \frac{4}{6} \implies\underline{\underline{ 0.67 }}\]

Her ser vi at alle nodene har lik closeness centrality på 0.67. Dette
vil si at ingen noder har kortere totale sti til alle nodene en noen
andre noder

    \hypertarget{closesness-centrality-for-stargraph}{%
\subsection*{Closesness centrality for
StarGraph:}\label{closesness-centrality-for-stargraph}}

\[\text{Formelen for closesness centrality for en node er:}\]

\[\frac{\text{Antall noder - 1}}{\text{Den totale avstanden fra den noden til alle andre noder}}\]

\textbf{Node 0}, den korteste veien til hver node:

Korteste vei fra 0 til 1 = {[}1{]} = 1

Korteste vei fra 0 til 2 = {[}2{]} = 1

Korteste vei fra 0 til 3 = {[}3{]} = 1

Korteste vei fra 0 til 4 = {[}4{]} = 1

Korteste vei fra 0 til 5 = {[}5{]} = 1

Den totale lengden: 5
\[\text{Closeness centrality} = \frac{6 - 1}{5} \implies \frac{5}{5} \implies\underline{\underline{ 1.0 }}\]

\textbf{Node 1}, den korteste veien til hver node:

Korteste vei fra 1 til 0 = {[}0{]} = 1

Korteste vei fra 1 til 2 = {[}0, 2{]} = 2

Korteste vei fra 1 til 3 = {[}0, 3{]} = 2

Korteste vei fra 1 til 4 = {[}0, 4{]} = 2

Korteste vei fra 1 til 5 = {[}0, 5{]} = 2

Den totale lengden: 9
\[\text{Closeness centrality} = \frac{6 - 1}{9} \implies \frac{5}{9} \implies\underline{\underline{ 0.56 }}\]

\textbf{Node 2}, den korteste veien til hver node:

Korteste vei fra 2 til 0 = {[}0{]} = 1

Korteste vei fra 2 til 1 = {[}0, 1{]} = 2

Korteste vei fra 2 til 3 = {[}0, 3{]} = 2

Korteste vei fra 2 til 4 = {[}0, 4{]} = 2

Korteste vei fra 2 til 5 = {[}0, 5{]} = 2

Den totale lengden: 9
\[\text{Closeness centrality} = \frac{6 - 1}{9} \implies \frac{5}{9} \implies\underline{\underline{ 0.56 }}\]

\textbf{Node 3}, den korteste veien til hver node:

Korteste vei fra 3 til 0 = {[}0{]} = 1

Korteste vei fra 3 til 1 = {[}0, 1{]} = 2

Korteste vei fra 3 til 2 = {[}0, 2{]} = 2

Korteste vei fra 3 til 4 = {[}0, 4{]} = 2

Korteste vei fra 3 til 5 = {[}0, 5{]} = 2

Den totale lengden: 9
\[\text{Closeness centrality} = \frac{6 - 1}{9} \implies \frac{5}{9} \implies\underline{\underline{ 0.56 }}\]

\textbf{Node 4}, den korteste veien til hver node:

Korteste vei fra 4 til 0 = {[}0{]} = 1

Korteste vei fra 4 til 1 = {[}0, 1{]} = 2

Korteste vei fra 4 til 2 = {[}0, 2{]} = 2

Korteste vei fra 4 til 3 = {[}0, 3{]} = 2

Korteste vei fra 4 til 5 = {[}0, 5{]} = 2

Den totale lengden: 9
\[\text{Closeness centrality} = \frac{6 - 1}{9} \implies \frac{5}{9} \implies\underline{\underline{ 0.56 }}\]

\textbf{Node 5}, den korteste veien til hver node:

Korteste vei fra 5 til 0 = {[}0{]} = 1

Korteste vei fra 5 til 1 = {[}0, 1{]} = 2

Korteste vei fra 5 til 2 = {[}0, 2{]} = 2

Korteste vei fra 5 til 3 = {[}0, 3{]} = 2

Korteste vei fra 5 til 4 = {[}0, 4{]} = 2

Den totale lengden: 9
\[\text{Closeness centrality} = \frac{6 - 1}{9} \implies \frac{5}{9} \implies\underline{\underline{ 0.56 }}\]

Her har node 0 høyest closeness centrality på 1, som vil si at den kan
nå alle de andre nodene direkte, som vi si at de har en avstand på 1.

    \hypertarget{closesness-centrality-for-treegraph}{%
\subsection*{Closesness centrality for
TreeGraph:}\label{closesness-centrality-for-treegraph}}

\[\text{Formelen for closesness centrality for en node er:}\]

\[\frac{\text{Antall noder - 1}}{\text{Den totale avstanden fra den noden til alle andre noder}}\]

\textbf{Node 0}, den korteste veien til hver node:

Korteste vei fra 0 til 1 = {[}1{]} = 1

Korteste vei fra 0 til 2 = {[}2{]} = 1

Korteste vei fra 0 til 3 = {[}1, 3{]} = 2

Korteste vei fra 0 til 4 = {[}1, 4{]} = 2

Korteste vei fra 0 til 5 = {[}2, 5{]} = 2

Korteste vei fra 0 til 6 = {[}2, 6{]} = 2

Den totale lengden: 10
\[\text{Closeness centrality} = \frac{7 - 1}{10} \implies \frac{6}{10} \implies\underline{\underline{ 0.6 }}\]

\textbf{Node 1}, den korteste veien til hver node:

Korteste vei fra 1 til 0 = {[}0{]} = 1

Korteste vei fra 1 til 2 = {[}0, 2{]} = 2

Korteste vei fra 1 til 3 = {[}3{]} = 1

Korteste vei fra 1 til 4 = {[}4{]} = 1

Korteste vei fra 1 til 5 = {[}0, 2, 5{]} = 3

Korteste vei fra 1 til 6 = {[}0, 2, 6{]} = 3

Den totale lengden: 11
\[\text{Closeness centrality} = \frac{7 - 1}{11} \implies \frac{6}{11} \implies\underline{\underline{ 0.55 }}\]

\textbf{Node 2}, den korteste veien til hver node:

Korteste vei fra 2 til 0 = {[}0{]} = 1

Korteste vei fra 2 til 1 = {[}0, 1{]} = 2

Korteste vei fra 2 til 3 = {[}0, 1, 3{]} = 3

Korteste vei fra 2 til 4 = {[}0, 1, 4{]} = 3

Korteste vei fra 2 til 5 = {[}5{]} = 1

Korteste vei fra 2 til 6 = {[}6{]} = 1

Den totale lengden: 11
\[\text{Closeness centrality} = \frac{7 - 1}{11} \implies \frac{6}{11} \implies\underline{\underline{ 0.55 }}\]

\textbf{Node 3}, den korteste veien til hver node:

Korteste vei fra 3 til 0 = {[}1, 0{]} = 2

Korteste vei fra 3 til 1 = {[}1{]} = 1

Korteste vei fra 3 til 2 = {[}1, 0, 2{]} = 3

Korteste vei fra 3 til 4 = {[}1, 4{]} = 2

Korteste vei fra 3 til 5 = {[}1, 0, 2, 5{]} = 4

Korteste vei fra 3 til 6 = {[}1, 0, 2, 6{]} = 4

Den totale lengden: 16
\[\text{Closeness centrality} = \frac{7 - 1}{16} \implies \frac{6}{16} \implies\underline{\underline{ 0.38 }}\]

\textbf{Node 4}, den korteste veien til hver node:

Korteste vei fra 4 til 0 = {[}1, 0{]} = 2

Korteste vei fra 4 til 1 = {[}1{]} = 1

Korteste vei fra 4 til 2 = {[}1, 0, 2{]} = 3

Korteste vei fra 4 til 3 = {[}1, 3{]} = 2

Korteste vei fra 4 til 5 = {[}1, 0, 2, 5{]} = 4

Korteste vei fra 4 til 6 = {[}1, 0, 2, 6{]} = 4

Den totale lengden: 16
\[\text{Closeness centrality} = \frac{7 - 1}{16} \implies \frac{6}{16} \implies\underline{\underline{ 0.38 }}\]

\textbf{Node 5}, den korteste veien til hver node:

Korteste vei fra 5 til 0 = {[}2, 0{]} = 2

Korteste vei fra 5 til 1 = {[}2, 0, 1{]} = 3

Korteste vei fra 5 til 2 = {[}2{]} = 1

Korteste vei fra 5 til 3 = {[}2, 0, 1, 3{]} = 4

Korteste vei fra 5 til 4 = {[}2, 0, 1, 4{]} = 4

Korteste vei fra 5 til 6 = {[}2, 6{]} = 2

Den totale lengden: 16
\[\text{Closeness centrality} = \frac{7 - 1}{16} \implies \frac{6}{16} \implies\underline{\underline{ 0.38 }}\]

\textbf{Node 6}, den korteste veien til hver node:

Korteste vei fra 6 til 0 = {[}2, 0{]} = 2

Korteste vei fra 6 til 1 = {[}2, 0, 1{]} = 3

Korteste vei fra 6 til 2 = {[}2{]} = 1

Korteste vei fra 6 til 3 = {[}2, 0, 1, 3{]} = 4

Korteste vei fra 6 til 4 = {[}2, 0, 1, 4{]} = 4

Korteste vei fra 6 til 5 = {[}2, 5{]} = 2

Den totale lengden: 16
\[\text{Closeness centrality} = \frac{7 - 1}{16} \implies \frac{6}{16} \implies\underline{\underline{ 0.38 }}\]

Her ser vi at node 0 har høyest closeness centrality på 0.6. Denne noden
er rotnoden i tregrafen, og den vil har kortest vei i snitt til en
tilfeldig node i grafen

    \hypertarget{closesness-centrality-for-meshgraph}{%
\subsection*{Closesness centrality for
MeshGraph:}\label{closesness-centrality-for-meshgraph}}

\[\text{Formelen for closesness centrality for en node er:}\]

\[\frac{\text{Antall noder - 1}}{\text{Den totale avstanden fra den noden til alle andre noder}}\]

\textbf{Node 0}, den korteste veien til hver node:

Korteste vei fra 0 til 1 = {[}1{]} = 1

Korteste vei fra 0 til 2 = {[}2{]} = 1

Korteste vei fra 0 til 3 = {[}3{]} = 1

Korteste vei fra 0 til 4 = {[}4{]} = 1

Den totale lengden: 4
\[\text{Closeness centrality} = \frac{5 - 1}{4} \implies \frac{4}{4} \implies\underline{\underline{ 1.0 }}\]

\textbf{Node 1}, den korteste veien til hver node:

Korteste vei fra 1 til 0 = {[}0{]} = 1

Korteste vei fra 1 til 2 = {[}2{]} = 1

Korteste vei fra 1 til 3 = {[}3{]} = 1

Korteste vei fra 1 til 4 = {[}4{]} = 1

Den totale lengden: 4
\[\text{Closeness centrality} = \frac{5 - 1}{4} \implies \frac{4}{4} \implies\underline{\underline{ 1.0 }}\]

\textbf{Node 2}, den korteste veien til hver node:

Korteste vei fra 2 til 0 = {[}0{]} = 1

Korteste vei fra 2 til 1 = {[}1{]} = 1

Korteste vei fra 2 til 3 = {[}3{]} = 1

Korteste vei fra 2 til 4 = {[}4{]} = 1

Den totale lengden: 4
\[\text{Closeness centrality} = \frac{5 - 1}{4} \implies \frac{4}{4} \implies\underline{\underline{ 1.0 }}\]

\textbf{Node 3}, den korteste veien til hver node:

Korteste vei fra 3 til 0 = {[}0{]} = 1

Korteste vei fra 3 til 1 = {[}1{]} = 1

Korteste vei fra 3 til 2 = {[}2{]} = 1

Korteste vei fra 3 til 4 = {[}4{]} = 1

Den totale lengden: 4
\[\text{Closeness centrality} = \frac{5 - 1}{4} \implies \frac{4}{4} \implies\underline{\underline{ 1.0 }}\]

\textbf{Node 4}, den korteste veien til hver node:

Korteste vei fra 4 til 0 = {[}0{]} = 1

Korteste vei fra 4 til 1 = {[}1{]} = 1

Korteste vei fra 4 til 2 = {[}2{]} = 1

Korteste vei fra 4 til 3 = {[}3{]} = 1

Den totale lengden: 4
\[\text{Closeness centrality} = \frac{5 - 1}{4} \implies \frac{4}{4} \implies\underline{\underline{ 1.0 }}\]

I denne grafen har alle nodene closeness centrality på 1, som vil si at
den kan nå alle de andre nodene direkte, som vi si at de har en avstand
på 1.

\emph{Merk at ingen av nodene i grafen har closeness centrality på 0,
for at det må til er avstanden til en node være uendelig som vil si at
det ikke finnes en vei til den noden, dette vil si at alle disse grafene
er sammenhengene og nodene kan nås fra hivlken som helst node}

    \hypertarget{oppgave-1.6}{%
\subsection*{Oppgave 1.6}\label{oppgave-1.6}}

Degree distribution kan vises med et histogram. Hva er degree
distribution og hvorfor er det logisk å se på det med et histogram?

For hver av grafene, lag et histogram over degree distribution og
kommenter hva du ser.

\begin{itemize}
\tightlist
\item
  Bruk metoden histogram() for å gjøre dette
\end{itemize}

VI forventer å se at studenten har skjønt hva et degree distribution
representerer og bruken av denne. I tillegg forventer vi et histogram
per graf, med en kort kommentar på hva dette histogrammet forteller om
grafen

    \begin{tcolorbox}[breakable, size=fbox, boxrule=1pt, pad at break*=1mm,colback=cellbackground, colframe=cellborder]
\prompt{In}{incolor}{19}{\boxspacing}
\begin{Verbatim}[commandchars=\\\{\}]
\PY{c+c1}{\PYZsh{}Lag alle histogrammene under}
\PY{n}{BussGraph}\PY{p}{(}\PY{l+m+mi}{5}\PY{p}{)}\PY{o}{.}\PY{n}{histogram}\PY{p}{(}\PY{p}{)}
\PY{n}{RingGraph}\PY{p}{(}\PY{l+m+mi}{5}\PY{p}{)}\PY{o}{.}\PY{n}{histogram}\PY{p}{(}\PY{p}{)}
\PY{n}{StarGraph}\PY{p}{(}\PY{l+m+mi}{5}\PY{p}{)}\PY{o}{.}\PY{n}{histogram}\PY{p}{(}\PY{p}{)}
\PY{n}{TreeGraph}\PY{p}{(}\PY{l+m+mi}{2}\PY{p}{,}\PY{l+m+mi}{2}\PY{p}{)}\PY{o}{.}\PY{n}{histogram}\PY{p}{(}\PY{p}{)}
\PY{n}{MeshGraph}\PY{p}{(}\PY{l+m+mi}{5}\PY{p}{)}\PY{o}{.}\PY{n}{histogram}\PY{p}{(}\PY{p}{)}\PY{p}{;}
\end{Verbatim}
\end{tcolorbox}

    \begin{center}
    \adjustimage{max size={0.9\linewidth}{0.9\paperheight}}{Oving3-revidert-v1_files/Oving3-revidert-v1_33_0.png}
    \end{center}
    { \hspace*{\fill} \\}
    
    \begin{center}
    \adjustimage{max size={0.9\linewidth}{0.9\paperheight}}{Oving3-revidert-v1_files/Oving3-revidert-v1_33_1.png}
    \end{center}
    { \hspace*{\fill} \\}
    
    \begin{center}
    \adjustimage{max size={0.9\linewidth}{0.9\paperheight}}{Oving3-revidert-v1_files/Oving3-revidert-v1_33_2.png}
    \end{center}
    { \hspace*{\fill} \\}
    
    \begin{center}
    \adjustimage{max size={0.9\linewidth}{0.9\paperheight}}{Oving3-revidert-v1_files/Oving3-revidert-v1_33_3.png}
    \end{center}
    { \hspace*{\fill} \\}
    
    \begin{center}
    \adjustimage{max size={0.9\linewidth}{0.9\paperheight}}{Oving3-revidert-v1_files/Oving3-revidert-v1_33_4.png}
    \end{center}
    { \hspace*{\fill} \\}
    
    \hypertarget{kommenter-foreldingen-av-node-degrees-som-vises-i-histogrammene-her}{%
\section*{Kommenter foreldingen av node degrees som vises i histogrammene
her:}\label{kommenter-foreldingen-av-node-degrees-som-vises-i-histogrammene-her}}

Det er logisk å se på et histogram av degree distribution fordi når det
blir et stort nettverk, kan det være vannskelig å få oversikten og se på
kjennetegnene nettverket har

\hypertarget{bussgraph}{%
\subsection*{BussGraph}\label{bussgraph}}

Her ser vi at histogrammet viser 2 noder med degree 1 og 3 med degree 2.
Siden den ikke er noen noder med degree 0, kan vi trygt si at grafen er
\emph{connected} som vil si at alle nodene kan nås fra hvilken som helst
node ved hjelp av nabonodene, dette gjelder også for de andre grafene
som vises.

\hypertarget{ringgraph}{%
\subsection*{RingGraph}\label{ringgraph}}

Her ser vi at alle nodene har degree 2, dette vil si at det er en
eularian path. og alle grafene kan vises \emph{topologisk} som er ring.

\hypertarget{stargraph}{%
\subsection*{StarGraph}\label{stargraph}}

Her ser vi at det er 5 noder med degree 1 og nøyaktig én node med degree
5. Eneste måten man kan tilfedstille dette på er at alle nodene med 1
degree har kanten sin med den med degree 5.

\hypertarget{treegraph}{%
\subsection*{TreeGraph}\label{treegraph}}

Her ser vi at det er 4 noder med degree 1, og i en \emph{TreeGraph}
betyr det at de nodene er såkalte \emph{``løvnoder''}.

Også ser vi at det er 1 node med degree 2, siden denne tregrafen
splitter i 2 hver gang, kan vi si at den har 2 \emph{barn} og ingen
\emph{forelder} som vi si at det er \emph{rotnoden}.

Det er også 2 noder med degree 3, som i en tregraf med 2 \emph{barn} vil
si at den har 2 \emph{barn} og en \emph{forelder}.

\hypertarget{meshgraph}{%
\subsection*{MeshGraph}\label{meshgraph}}

Her er det 5 noder av degree 4, som vil si at alle nodene har en kant
til alle andre noder minus seg selv

    \hypertarget{del-2-strukturanalyse}{%
\section*{Del 2: Strukturanalyse}\label{del-2-strukturanalyse}}

I denne delen skal vi introdusere flere grafstrukturer. Disse
grafstrukturene er mer komplekse, men gjenspeiler bedre relle nettverk.
Et tips her er å bruke seed=student\_seed for å få samme graf her gang.

    \hypertarget{oppgave-2.1}{%
\subsection*{Oppgave 2.1}\label{oppgave-2.1}}

Lag følgende 4 grafer, alle med 100 noder:

\begin{verbatim}
Graf 1: En Barabasi Albert graf med parameter m=1
Graf 2: En Barabasi Albert graf med parameter m=2
Graf 3: En Watts Strogatz graf med parametre k=2 og p=0.1
Graf 4: En Watts Strogatz graf med parametre k=4 og p=0.1
\end{verbatim}

For hver av grafene, tegn de og lag et histogram over degree
distribution.

Her forventer vi at de fire grafene blir konstruert og printet. Vi
forventer også å se et histogram for hver graf

    \begin{tcolorbox}[breakable, size=fbox, boxrule=1pt, pad at break*=1mm,colback=cellbackground, colframe=cellborder]
\prompt{In}{incolor}{159}{\boxspacing}
\begin{Verbatim}[commandchars=\\\{\}]
\PY{c+c1}{\PYZsh{}Kode her}
\PY{n}{b1} \PY{o}{=} \PY{n}{BarabasiAlbert}\PY{p}{(}\PY{n}{n}\PY{o}{=} \PY{l+m+mi}{100}\PY{p}{,} \PY{n}{m}\PY{o}{=}\PY{l+m+mi}{1}\PY{p}{,} \PY{n}{seed}\PY{o}{=}\PY{n}{student\PYZus{}seed}\PY{p}{)}
\PY{n}{b2} \PY{o}{=} \PY{n}{BarabasiAlbert}\PY{p}{(}\PY{n}{n}\PY{o}{=} \PY{l+m+mi}{100}\PY{p}{,} \PY{n}{m}\PY{o}{=}\PY{l+m+mi}{2}\PY{p}{,} \PY{n}{seed}\PY{o}{=}\PY{n}{student\PYZus{}seed}\PY{p}{)}

\PY{n}{w1} \PY{o}{=} \PY{n}{WattsStrogatz}\PY{p}{(}\PY{n}{n}\PY{o}{=}\PY{l+m+mi}{100}\PY{p}{,} \PY{n}{k}\PY{o}{=}\PY{l+m+mi}{2}\PY{p}{,} \PY{n}{p}\PY{o}{=}\PY{l+m+mf}{0.1}\PY{p}{,} \PY{n}{seed}\PY{o}{=}\PY{n}{student\PYZus{}seed}\PY{p}{)}
\PY{n}{w2} \PY{o}{=} \PY{n}{WattsStrogatz}\PY{p}{(}\PY{n}{n}\PY{o}{=}\PY{l+m+mi}{100}\PY{p}{,} \PY{n}{k}\PY{o}{=}\PY{l+m+mi}{4}\PY{p}{,} \PY{n}{p}\PY{o}{=}\PY{l+m+mf}{0.1}\PY{p}{,} \PY{n}{seed}\PY{o}{=}\PY{n}{student\PYZus{}seed}\PY{p}{)}

\PY{n}{b1}\PY{o}{.}\PY{n}{histogram}\PY{p}{(}\PY{p}{)}
\PY{n}{b2}\PY{o}{.}\PY{n}{histogram}\PY{p}{(}\PY{p}{)}
\PY{n}{w1}\PY{o}{.}\PY{n}{histogram}\PY{p}{(}\PY{p}{)}
\PY{n}{w2}\PY{o}{.}\PY{n}{histogram}\PY{p}{(}\PY{p}{)}

\PY{n}{b1}\PY{o}{.}\PY{n}{draw}\PY{p}{(}\PY{p}{)}
\PY{n}{b2}\PY{o}{.}\PY{n}{draw}\PY{p}{(}\PY{p}{)}
\PY{n}{w1}\PY{o}{.}\PY{n}{draw}\PY{p}{(}\PY{p}{)}
\PY{n}{w2}\PY{o}{.}\PY{n}{draw}\PY{p}{(}\PY{p}{)}
\end{Verbatim}
\end{tcolorbox}

    \begin{center}
    \adjustimage{max size={0.9\linewidth}{0.9\paperheight}}{Oving3-revidert-v1_files/Oving3-revidert-v1_37_0.png}
    \end{center}
    { \hspace*{\fill} \\}
    
    \begin{center}
    \adjustimage{max size={0.9\linewidth}{0.9\paperheight}}{Oving3-revidert-v1_files/Oving3-revidert-v1_37_1.png}
    \end{center}
    { \hspace*{\fill} \\}
    
    \begin{center}
    \adjustimage{max size={0.9\linewidth}{0.9\paperheight}}{Oving3-revidert-v1_files/Oving3-revidert-v1_37_2.png}
    \end{center}
    { \hspace*{\fill} \\}
    
    \begin{center}
    \adjustimage{max size={0.9\linewidth}{0.9\paperheight}}{Oving3-revidert-v1_files/Oving3-revidert-v1_37_3.png}
    \end{center}
    { \hspace*{\fill} \\}
    
    \begin{center}
    \adjustimage{max size={0.9\linewidth}{0.9\paperheight}}{Oving3-revidert-v1_files/Oving3-revidert-v1_37_4.png}
    \end{center}
    { \hspace*{\fill} \\}
    
    \begin{center}
    \adjustimage{max size={0.9\linewidth}{0.9\paperheight}}{Oving3-revidert-v1_files/Oving3-revidert-v1_37_5.png}
    \end{center}
    { \hspace*{\fill} \\}
    
    \begin{center}
    \adjustimage{max size={0.9\linewidth}{0.9\paperheight}}{Oving3-revidert-v1_files/Oving3-revidert-v1_37_6.png}
    \end{center}
    { \hspace*{\fill} \\}
    
    \begin{center}
    \adjustimage{max size={0.9\linewidth}{0.9\paperheight}}{Oving3-revidert-v1_files/Oving3-revidert-v1_37_7.png}
    \end{center}
    { \hspace*{\fill} \\}
    
    \hypertarget{oppgave-2.1.1}{%
\subsection*{Oppgave 2.1.1}\label{oppgave-2.1.1}}

Kommenter histogrammene over på hensyn av eventuelle styrker og
sårbarheter de forskjellige grafene har

Her forventer vi en sammenhengende tekst der man ser på likheter og
forskjeller mellom de forskjellige grafene, med fokus på styrker og
sårbarheter

    \hypertarget{forklaring-her}{%
\section*{Forklaring her:}\label{forklaring-her}}

\hypertarget{barabasialbert-m1}{%
\subsection*{BarabasiAlbert (m=1):}\label{barabasialbert-m1}}

Her ser vi at at det er klart flest noder med 1 degree, Også er det noen
får noder med svært høy degree

\hypertarget{svakheter}{%
\subsubsection*{Svakheter:}\label{svakheter}}

Nodene med mange kanter vil ha høy \emph{betweennes centrality}, fordi
den korteste veien fra en node til en annen har høy sannylighet til å
være via den.

Dette vil si at den noden vil motta høy trafikk og om den blir fjernet
vil mange noder ikke kunne snakke med hverandre

\hypertarget{styrker}{%
\subsubsection*{Styrker:}\label{styrker}}

Siden det er lett å finne disse nodene med høy degree kan man i et
nettverk gjøre disse nodene robuste og rakse slik at det har kapasiteten
til å håndtere trafikken mellom nodene

\hypertarget{barabasialbert-m2}{%
\subsection*{BarabasiAlbert (m=2):}\label{barabasialbert-m2}}

Her ser vi at det ikke er noen noder med lavere degree enn 2. Det er
fler noder med fler kanter enn på BA (m=1) som vil gjøre at trafikken
blir mer fordelt gjennom de nodene. Det er også mange fler degrees i
denne grafen enn i BA(m=1)

\hypertarget{svakheter-1}{%
\subsubsection*{Svakheter:}\label{svakheter-1}}

Det er veldig mange degrees i grafen som gjør at det kan bli kostbart og
kontruere og vedlikeholde

\hypertarget{styrker-1}{%
\subsubsection*{Styrker:}\label{styrker-1}}

Det vil være større sansylighet for at det er kortere veier fra en node
til en annen enn i BA (m=2) som vil redusere trafikk

\hypertarget{wattsstrogatz-k2-p0.1}{%
\subsection*{WattsStrogatz (k=2, p=0.1):}\label{wattsstrogatz-k2-p0.1}}

Her ser vi at alle nodene har degree mellom 1 og 3, som vil si at det er
ingen noder som har høy degree

\hypertarget{svakheter-2}{%
\subsubsection*{Svakheter:}\label{svakheter-2}}

Man kan se at grafen er bygd opp en \emph{ringGrafer} og
\emph{bussgrafer} som går ut av ringen, om man fjerner den en node i
ronggrafen som er koblet opp til en buss, vi hele bussen miste
\emph{kontakten} med resten av grafen \#\#\# Styrker: Det er veldig få
kanter, fordi ingen har høy kantverdi dette vil si at det ikke er så
kostbart å konstruere og vedlikeholde

\hypertarget{wattsstrogatz-k4-p0.1}{%
\subsection*{WattsStrogatz (k=4, p=0.1):}\label{wattsstrogatz-k4-p0.1}}

Her ser vi at det er ingen noder med degree 1 og veldig få med degreee
2,

Det er også ingen noder med høyere grad enn 5, med liket til WS (k=2)
som kun har 4 \#\#\# Svakheter: Utrolig mange kanter, dette kan gjøre
det veldig kostbart å kostruere og vedlikeholde \#\#\# Styrker: Ingen
noder med høyere enn 5 kanter, dette gjør det vannskelig for hackere å
målrette seg mot én spesiell kant

    \hypertarget{oppgave-2.1.2}{%
\subsection*{Oppgave 2.1.2}\label{oppgave-2.1.2}}

Konstruer et nettverk bestående av mellom 9 og 15 noder, med en node som
har høyest betweenness centrality, og en av de laveste degree centrality

Her forventer vi at studenten holder seg innenfor grensen på 9-15 noder,
samt har konstruert en graf der en node har høyest betweenness
centrality, men også en av de laveste degree centralitiene

    \begin{tcolorbox}[breakable, size=fbox, boxrule=1pt, pad at break*=1mm,colback=cellbackground, colframe=cellborder]
\prompt{In}{incolor}{11}{\boxspacing}
\begin{Verbatim}[commandchars=\\\{\}]
\PY{c+c1}{\PYZsh{}Kode her}

\PY{n}{graph} \PY{o}{=} \PY{n}{Graph}\PY{p}{(}\PY{p}{)}
\PY{n}{graph}\PY{o}{.}\PY{n}{add\PYZus{}node}\PY{p}{(}\PY{l+s+s2}{\PYZdq{}}\PY{l+s+s2}{TARGET}\PY{l+s+s2}{\PYZdq{}}\PY{p}{)}

\PY{k}{for} \PY{n}{i} \PY{o+ow}{in} \PY{n+nb}{range}\PY{p}{(}\PY{l+m+mi}{2}\PY{p}{)}\PY{p}{:}
    \PY{n}{mesh} \PY{o}{=} \PY{n}{MeshGraph}\PY{p}{(}\PY{n+nb}{range}\PY{p}{(}\PY{n}{i}\PY{o}{*}\PY{l+m+mi}{6}\PY{p}{,} \PY{p}{(}\PY{n}{i}\PY{o}{+}\PY{l+m+mi}{1}\PY{p}{)}\PY{o}{*}\PY{l+m+mi}{6}\PY{p}{)}\PY{p}{)}
    \PY{n}{graph}\PY{o}{.}\PY{n}{add\PYZus{}nodes\PYZus{}from}\PY{p}{(}\PY{n}{mesh}\PY{o}{.}\PY{n}{nodes}\PY{p}{(}\PY{p}{)}\PY{p}{)}
    \PY{n}{graph}\PY{o}{.}\PY{n}{add\PYZus{}edges\PYZus{}from}\PY{p}{(}\PY{n}{mesh}\PY{o}{.}\PY{n}{edges}\PY{p}{(}\PY{p}{)}\PY{p}{)}
    \PY{n}{graph}\PY{o}{.}\PY{n}{add\PYZus{}edges\PYZus{}from}\PY{p}{(}\PY{p}{[}\PY{p}{(}\PY{l+s+s2}{\PYZdq{}}\PY{l+s+s2}{TARGET}\PY{l+s+s2}{\PYZdq{}}\PY{p}{,}\PY{n}{i}\PY{o}{*}\PY{l+m+mi}{6}\PY{p}{)}\PY{p}{,}\PY{p}{(}\PY{l+s+s2}{\PYZdq{}}\PY{l+s+s2}{TARGET}\PY{l+s+s2}{\PYZdq{}}\PY{p}{,} \PY{n}{i}\PY{o}{*}\PY{l+m+mi}{6} \PY{o}{+} \PY{l+m+mi}{1}\PY{p}{)}\PY{p}{]}\PY{p}{)}

\PY{n}{graph}\PY{o}{.}\PY{n}{draw\PYZus{}betweenness\PYZus{}centrality}\PY{p}{(}\PY{p}{)}
\PY{n}{graph}\PY{o}{.}\PY{n}{draw\PYZus{}degree\PYZus{}centrality}\PY{p}{(}\PY{p}{)}
\end{Verbatim}
\end{tcolorbox}

    \begin{center}
    \adjustimage{max size={0.9\linewidth}{0.9\paperheight}}{Oving3-revidert-v1_files/Oving3-revidert-v1_41_0.png}
    \end{center}
    { \hspace*{\fill} \\}
    
    \begin{center}
    \adjustimage{max size={0.9\linewidth}{0.9\paperheight}}{Oving3-revidert-v1_files/Oving3-revidert-v1_41_1.png}
    \end{center}
    { \hspace*{\fill} \\}
    
    \hypertarget{oppgave-2.2}{%
\subsection*{Oppgave 2.2}\label{oppgave-2.2}}

    \begin{tcolorbox}[breakable, size=fbox, boxrule=1pt, pad at break*=1mm,colback=cellbackground, colframe=cellborder]
\prompt{In}{incolor}{14}{\boxspacing}
\begin{Verbatim}[commandchars=\\\{\}]
\PY{n}{networkURL} \PY{o}{=} \PY{n}{get\PYZus{}network}\PY{p}{(}\PY{n}{student\PYZus{}seed}\PY{p}{)}
\PY{c+c1}{\PYZsh{} Du kan bruke variabelen networkURL som parameter i RealNetworkGraph}
\PY{c+c1}{\PYZsh{} Ved å bruke student\PYZus{}seed vil metoden hente samme nettverk hver gang}
\end{Verbatim}
\end{tcolorbox}

    \begin{Verbatim}[commandchars=\\\{\}]
You will analyze the Garr201110 network.
Your network graph file is http://www.topology-zoo.org/files/Garr201110.graphml

    \end{Verbatim}

    I denne oppgaven skal vi analysere et ekte nettverk. Klassen
RealNetworkGraph() tar inn en url av en fil med filtypen .graphml. Lag
et objekt for nettverket du får i koden over og tegn det. Filene som kan
analyseres finnes på nettsiden www.topology-zoo.org/dataset.html.
Funksjonen over henter ut en ``.graphml'' fil du kan bruke, du kan
eventuelt hente den ut selv og se på andre ved å se på nettsiden.

Her forventer vi å se at studenten klarer å hente ut grafen som er blitt
tildelt, konstruerer den og viser den frem i cellen nedenfor.

    \begin{tcolorbox}[breakable, size=fbox, boxrule=1pt, pad at break*=1mm,colback=cellbackground, colframe=cellborder]
\prompt{In}{incolor}{15}{\boxspacing}
\begin{Verbatim}[commandchars=\\\{\}]
\PY{c+c1}{\PYZsh{} Konstruer og tegn nettverket her}
\PY{n}{rng} \PY{o}{=} \PY{n}{RealNetworkGraph}\PY{p}{(}\PY{n}{networkURL}\PY{p}{)}
\PY{n}{rng}\PY{o}{.}\PY{n}{draw}\PY{p}{(}\PY{p}{)}
\end{Verbatim}
\end{tcolorbox}

    \begin{center}
    \adjustimage{max size={0.9\linewidth}{0.9\paperheight}}{Oving3-revidert-v1_files/Oving3-revidert-v1_45_0.png}
    \end{center}
    { \hspace*{\fill} \\}
    
    \hypertarget{oppgave-2.2.1}{%
\subsection*{Oppgave 2.2.1}\label{oppgave-2.2.1}}

Plott et histogram over degree distribution for nettverket over. Hva
forteller histogrammet deg?

Her forventer vi å se et histogram og en tekst om hva histogrammet
forteller. Her kan det være lurt å blande inn histogrammer fra tidligere
oppgaver, og bruke disse til å drøfte det reelle nettverket.

    \begin{tcolorbox}[breakable, size=fbox, boxrule=1pt, pad at break*=1mm,colback=cellbackground, colframe=cellborder]
\prompt{In}{incolor}{17}{\boxspacing}
\begin{Verbatim}[commandchars=\\\{\}]
\PY{c+c1}{\PYZsh{}Kode}
\PY{n}{rng}\PY{o}{.}\PY{n}{histogram}\PY{p}{(}\PY{p}{)}\PY{p}{;}
\end{Verbatim}
\end{tcolorbox}

    \begin{center}
    \adjustimage{max size={0.9\linewidth}{0.9\paperheight}}{Oving3-revidert-v1_files/Oving3-revidert-v1_47_0.png}
    \end{center}
    { \hspace*{\fill} \\}
    
    \hypertarget{forklaring}{%
\subsection*{Forklaring}\label{forklaring}}

Dette histogrammet likner på en Barabasi Albert graf med m=2 (se oppg
2.1), fordi den har noen noder med mange node degrees. Disse nodene er
sentrale for at nettverket skal kjøre.

Vi ser også at det er de fleste endeSystemene, kun er koblet opp til en
sentral node. Dette vil si at hvis den sentrale noden feiler, så vil
ikke disse endesystemene kunne kommunistere med nettverket

    \hypertarget{oppgave-2.2.2}{%
\subsection*{Oppgave 2.2.2}\label{oppgave-2.2.2}}

Under er det oppgitt en funksjon for å gi ut hvilke noder som er
viktigst med de tre funksjonalitetene. Bruk funksjonen på grafen. Er
noen noder viktig i flere av sentralitetene? Hvorfor?

Her forventer vi at studenten klarer å bruke metoden som er gitt, til å
finne de viktigste nodene i hver kategori. Deretter forventer vi at
studenten klarer å finne noder som er viktige i flere kategorier, og
drøfter kort rundt dette.

    \begin{tcolorbox}[breakable, size=fbox, boxrule=1pt, pad at break*=1mm,colback=cellbackground, colframe=cellborder]
\prompt{In}{incolor}{12}{\boxspacing}
\begin{Verbatim}[commandchars=\\\{\}]
\PY{k}{def} \PY{n+nf}{get\PYZus{}centrality\PYZus{}table}\PY{p}{(}\PY{n}{graph}\PY{p}{)}\PY{p}{:}
    \PY{n}{deg}\PY{o}{=} \PY{n}{graph}\PY{o}{.}\PY{n}{degree\PYZus{}centrality}\PY{p}{(}\PY{p}{)}
    \PY{n+nb+bp}{cls} \PY{o}{=} \PY{n}{graph}\PY{o}{.}\PY{n}{closeness\PYZus{}centrality}\PY{p}{(}\PY{p}{)}
    \PY{n}{betw} \PY{o}{=} \PY{n}{graph}\PY{o}{.}\PY{n}{betweenness\PYZus{}centrality}\PY{p}{(}\PY{p}{)}
    \PY{n}{lst} \PY{o}{=} \PY{p}{[}\PY{n}{deg}\PY{p}{,}\PY{n+nb+bp}{cls}\PY{p}{,}\PY{n}{betw}\PY{p}{]}
    \PY{k}{for} \PY{n}{i}\PY{p}{,}\PY{n}{obj} \PY{o+ow}{in} \PY{n+nb}{enumerate}\PY{p}{(}\PY{n}{lst}\PY{p}{)}\PY{p}{:}
        \PY{n}{values} \PY{o}{=} \PY{p}{[}\PY{p}{(}\PY{n}{graph}\PY{o}{.}\PY{n}{nodes}\PY{p}{[}\PY{n}{k}\PY{p}{]}\PY{p}{[}\PY{l+s+s2}{\PYZdq{}}\PY{l+s+s2}{label}\PY{l+s+s2}{\PYZdq{}}\PY{p}{]}\PY{p}{,}\PY{n}{v}\PY{p}{)} \PY{k}{for} \PY{n}{k}\PY{p}{,} \PY{n}{v} \PY{o+ow}{in} \PY{n+nb}{sorted}\PY{p}{(}\PY{n}{obj}\PY{o}{.}\PY{n}{items}\PY{p}{(}\PY{p}{)}\PY{p}{,} \PY{n}{key}\PY{o}{=}\PY{k}{lambda} \PY{n}{item}\PY{p}{:} \PY{n}{item}\PY{p}{[}\PY{l+m+mi}{1}\PY{p}{]}\PY{p}{)}\PY{p}{]}
        \PY{n}{values}\PY{o}{.}\PY{n}{reverse}\PY{p}{(}\PY{p}{)}
        \PY{n}{lst}\PY{p}{[}\PY{n}{i}\PY{p}{]} \PY{o}{=} \PY{n}{values}
    \PY{n+nb}{print}\PY{p}{(}\PY{l+s+s2}{\PYZdq{}}\PY{l+s+se}{\PYZbs{}n}\PY{l+s+s2}{Centrality Indexer,}\PY{l+s+se}{\PYZbs{}n}\PY{l+s+s2}{\PYZhy{}sortert i synkende rekkefølge}\PY{l+s+se}{\PYZbs{}n}\PY{l+s+s2}{\PYZdq{}}\PY{p}{)}
    \PY{n+nb}{print}\PY{p}{(}\PY{l+s+s2}{\PYZdq{}}\PY{l+s+se}{\PYZbs{}033}\PY{l+s+s2}{[1m}\PY{l+s+s2}{\PYZdq{}} \PY{o}{+} \PY{l+s+s2}{\PYZdq{}}\PY{l+s+s2}{Degree}\PY{l+s+s2}{\PYZdq{}}\PY{o}{.}\PY{n}{ljust}\PY{p}{(}\PY{l+m+mi}{20}\PY{p}{)} \PY{o}{+} \PY{l+s+s2}{\PYZdq{}}\PY{l+s+s2}{Closeness}\PY{l+s+s2}{\PYZdq{}}\PY{o}{.}\PY{n}{ljust}\PY{p}{(}\PY{l+m+mi}{20}\PY{p}{)} \PY{o}{+} \PY{l+s+s2}{\PYZdq{}}\PY{l+s+s2}{Betweenes}\PY{l+s+s2}{\PYZdq{}}\PY{o}{.}\PY{n}{ljust}\PY{p}{(}\PY{l+m+mi}{20}\PY{p}{)} \PY{o}{+} \PY{l+s+s1}{\PYZsq{}}\PY{l+s+se}{\PYZbs{}033}\PY{l+s+s1}{[0m}\PY{l+s+s1}{\PYZsq{}}\PY{p}{)}
    \PY{k}{for} \PY{n}{deg}\PY{p}{,}\PY{n+nb+bp}{cls}\PY{p}{,}\PY{n}{betw} \PY{o+ow}{in} \PY{n+nb}{zip}\PY{p}{(}\PY{o}{*}\PY{n}{lst}\PY{p}{)}\PY{p}{:}
        \PY{n+nb}{print}\PY{p}{(}\PY{n}{deg}\PY{p}{[}\PY{l+m+mi}{0}\PY{p}{]}\PY{o}{.}\PY{n}{ljust}\PY{p}{(}\PY{l+m+mi}{20}\PY{p}{)} \PY{o}{+} \PY{n+nb+bp}{cls}\PY{p}{[}\PY{l+m+mi}{0}\PY{p}{]}\PY{o}{.}\PY{n}{ljust}\PY{p}{(}\PY{l+m+mi}{20}\PY{p}{)} \PY{o}{+} \PY{n}{betw}\PY{p}{[}\PY{l+m+mi}{0}\PY{p}{]}\PY{o}{.}\PY{n}{ljust}\PY{p}{(}\PY{l+m+mi}{20}\PY{p}{)}\PY{p}{)}
\end{Verbatim}
\end{tcolorbox}

    \begin{tcolorbox}[breakable, size=fbox, boxrule=1pt, pad at break*=1mm,colback=cellbackground, colframe=cellborder]
\prompt{In}{incolor}{18}{\boxspacing}
\begin{Verbatim}[commandchars=\\\{\}]
\PY{c+c1}{\PYZsh{}Kode her}
\PY{n}{get\PYZus{}centrality\PYZus{}table}\PY{p}{(}\PY{n}{rng}\PY{p}{)}
\end{Verbatim}
\end{tcolorbox}

    \begin{Verbatim}[commandchars=\\\{\}]

Centrality Indexer,
-sortert i synkende rekkefølge

\textbf{Degree              Closeness           Betweenes           }
MI-2                RM-2                RM-2
RM-2                BO                  MI-2
MI-1                MI-2                BO
BO                  MI-1                CT
BA                  NA                  NA
PD                  RM-1                MI-1
CT                  BA                  BA
RM-1                PD                  RM-1
NA                  CA-1                PD
PI                  AN                  FI
CO                  PI                  CA-1
FI                  FRA                 PI
CA-1                CO                  AN
AN                  MI-3                TO
TO                  TN                  Pv
MI-3                AQ-1                PD-2
Pv                  FI                  PA
AQ                  FUC                 CS
FRA                 PG                  MI-3
GE                  NAMEX               CO
MI-4                TO                  TN
TN                  CT                  AQ-1
PD-2                GE                  GE
PA                  Fe                  AQ
CS                  BO-3                TS-1
TS-1                Pv                  Ur
AQ-1                MI-4                Fe
Ur                  Google              TO-PIX
Fe                  Level 3             BS
TO-PIX              MIX                 GEANT
BS                  GEANT               Google
GEANT               TS-1                FG
Google              GEANT               LE
FG                  Global Crossing     MT
LE                  Svizzera            FRA
MT                  CB                  CB
CB                  SA                  PZ
PZ                  AQ                  SA
SA                  PD-2                MI-4
VE                  FG                  VE
Fi                  LE                  Fi
FUC                 MT                  FUC
BO-3                PZ                  BO-3
Level 3             VSIX                Level 3
MIX                 SS                  MIX
EUMED CONNECT       CA                  EUMED CONNECT
ME                  Ur                  ME
VSIX                BS                  VSIX
TIX                 Fi                  TIX
CZ                  TIX                 CZ
PG                  PA                  PG
Pv-1                CS                  Pv-1
PA-2                TO-PIX              PA-2
SS                  EUMED CONNECT       SS
NAMEX               ME                  NAMEX
GEANT               Pv-1                GEANT
Global Crossing     VE                  Global Crossing
CA                  CZ                  CA
Svizzera            PA-2                Svizzera
    \end{Verbatim}

    \hypertarget{forklaring}{%
\subsection*{Forklaring}\label{forklaring}}

\textbf{MI-2} noden scorer høyest på degree centrality, fordi den har 13
noder, som er flest noden i grafen

\textbf{RM-2} node har høyest closeness- og betweennes-verdi og den har
nest høyest degree-verdi

\textbf{RM-2}:

\begin{verbatim}
Den har høyest betweenness-verdi fordi kantene til noden strekker over til kanter er "langt" unna hverandre,
sett bort ifra den selv. Dette vil si at de andre kantene som skal over til en som egentlig er "langt" unna, kan
bruke RM-2 noden for få kortest mulig sti.

Den har høyest closeness-verdi også av samme grunn som betweenness som gjør at den korteste veien til alle noder
blir kort fordi den slipper å ta den vanlige veien.

Den har nest høyest degree, som kommer av at den har 11 kanter som er nest mest i denne grafen. dette gjør noden sentral.
\end{verbatim}

    \hypertarget{oppgave-2.3}{%
\subsection*{Oppgave 2.3}\label{oppgave-2.3}}

ConstructedGraph() simulerer et reelt nettverk, bestående av et
kjernenett med grid-struktur, et regionalnett og et tettbebygd
aksessnett. Bruk klassen og tegn grafen.

For hver av de tre centralitiene, finn de mest sentrale nodene og tegn
de. Hvor ligger de viktigste nodene, og hvordan ville du beskrevet
robustheten til regionalnettet?

Her forventer vi at studenten klarer å bruke ConstructedGraph() til å
printe ut et reelt nettverk, og at studenten skjønner hvilke deler dette
nettverket er bygget opp av. Deretter forventer vi at studenten viser de
mest sentrale nodene innenfor hver kategori, forteller hvor i grafen de
ligger, og drøfter robustheten til regionalnettet.

    \begin{tcolorbox}[breakable, size=fbox, boxrule=1pt, pad at break*=1mm,colback=cellbackground, colframe=cellborder]
\prompt{In}{incolor}{10}{\boxspacing}
\begin{Verbatim}[commandchars=\\\{\}]
\PY{c+c1}{\PYZsh{}Kode her}
\PY{n}{cg} \PY{o}{=} \PY{n}{ConstructedGraph}\PY{p}{(}\PY{p}{)}

\PY{n}{deg}\PY{o}{=} \PY{n}{cg}\PY{o}{.}\PY{n}{degree\PYZus{}centrality}\PY{p}{(}\PY{p}{)}
\PY{n}{betw} \PY{o}{=} \PY{n}{cg}\PY{o}{.}\PY{n}{betweenness\PYZus{}centrality}\PY{p}{(}\PY{p}{)}
\PY{n+nb+bp}{cls} \PY{o}{=} \PY{n}{cg}\PY{o}{.}\PY{n}{closeness\PYZus{}centrality}\PY{p}{(}\PY{p}{)}


\PY{c+c1}{\PYZsh{} get the node(s) with the max value, there might be more nodes with the same max value}
\PY{n}{mx\PYZus{}deg} \PY{o}{=} \PY{n+nb}{max}\PY{p}{(}\PY{n}{deg}\PY{o}{.}\PY{n}{values}\PY{p}{(}\PY{p}{)}\PY{p}{)}
\PY{n}{max\PYZus{}deg} \PY{o}{=} \PY{p}{[}\PY{n}{k} \PY{k}{for} \PY{n}{k}\PY{p}{,} \PY{n}{v} \PY{o+ow}{in} \PY{n}{deg}\PY{o}{.}\PY{n}{items}\PY{p}{(}\PY{p}{)} \PY{k}{if} \PY{n}{v} \PY{o}{==} \PY{n}{mx\PYZus{}deg}\PY{p}{]}
\PY{n}{cg}\PY{o}{.}\PY{n}{mark\PYZus{}nodes}\PY{p}{(}\PY{n}{max\PYZus{}deg}\PY{p}{)}

\PY{n}{mx\PYZus{}betw} \PY{o}{=} \PY{n+nb}{max}\PY{p}{(}\PY{n}{betw}\PY{o}{.}\PY{n}{values}\PY{p}{(}\PY{p}{)}\PY{p}{)}
\PY{n}{max\PYZus{}betw} \PY{o}{=} \PY{p}{[}\PY{n}{k} \PY{k}{for} \PY{n}{k}\PY{p}{,} \PY{n}{v} \PY{o+ow}{in} \PY{n}{betw}\PY{o}{.}\PY{n}{items}\PY{p}{(}\PY{p}{)} \PY{k}{if} \PY{n}{v} \PY{o}{==} \PY{n}{mx\PYZus{}betw}\PY{p}{]}
\PY{n}{cg}\PY{o}{.}\PY{n}{mark\PYZus{}nodes}\PY{p}{(}\PY{n}{max\PYZus{}betw}\PY{p}{)}

\PY{n}{mx\PYZus{}cls} \PY{o}{=} \PY{n+nb}{max}\PY{p}{(}\PY{n+nb+bp}{cls}\PY{o}{.}\PY{n}{values}\PY{p}{(}\PY{p}{)}\PY{p}{)}
\PY{n}{max\PYZus{}cls} \PY{o}{=} \PY{p}{[}\PY{n}{k} \PY{k}{for} \PY{n}{k}\PY{p}{,} \PY{n}{v} \PY{o+ow}{in} \PY{n+nb+bp}{cls}\PY{o}{.}\PY{n}{items}\PY{p}{(}\PY{p}{)} \PY{k}{if} \PY{n}{v} \PY{o}{==} \PY{n}{mx\PYZus{}cls}\PY{p}{]}
\PY{n}{cg}\PY{o}{.}\PY{n}{mark\PYZus{}nodes}\PY{p}{(}\PY{n}{max\PYZus{}cls}\PY{p}{)}

\PY{n}{maxes} \PY{o}{=} \PY{p}{\PYZob{}}\PY{p}{\PYZcb{}}
\PY{n}{maxes}\PY{p}{[}\PY{l+s+s2}{\PYZdq{}}\PY{l+s+s2}{Degree centrality}\PY{l+s+s2}{\PYZdq{}}\PY{p}{]} \PY{o}{=} \PY{n}{max\PYZus{}deg}
\PY{n}{maxes}\PY{p}{[}\PY{l+s+s2}{\PYZdq{}}\PY{l+s+s2}{Betweenness centrality}\PY{l+s+s2}{\PYZdq{}}\PY{p}{]} \PY{o}{=} \PY{n}{max\PYZus{}betw}
\PY{n}{maxes}\PY{p}{[}\PY{l+s+s2}{\PYZdq{}}\PY{l+s+s2}{Closeness centrality}\PY{l+s+s2}{\PYZdq{}}\PY{p}{]} \PY{o}{=} \PY{n}{max\PYZus{}cls}
\PY{k}{for} \PY{n}{name}\PY{p}{,} \PY{n}{maks} \PY{o+ow}{in} \PY{n}{maxes}\PY{o}{.}\PY{n}{items}\PY{p}{(}\PY{p}{)}\PY{p}{:}
    \PY{n+nb}{print}\PY{p}{(}\PY{l+s+s2}{\PYZdq{}}\PY{l+s+s2}{Mest sentrale node for}\PY{l+s+s2}{\PYZdq{}}\PY{p}{,} \PY{n}{name}\PY{p}{,} \PY{l+s+s2}{\PYZdq{}}\PY{l+s+s2}{er:}\PY{l+s+s2}{\PYZdq{}}\PY{p}{,} \PY{n}{maks}\PY{p}{)}
\end{Verbatim}
\end{tcolorbox}

    \begin{Verbatim}[commandchars=\\\{\}]
Mest sentrale node for Degree centrality er: ['d0', 'aa0', 'ab0', 'ac0', 'ad0',
'ba0', 'bb0', 'bc0', 'bd0']
Mest sentrale node for Betweenness centrality er: ['d0']
Mest sentrale node for Closeness centrality er: ['core4']
    \end{Verbatim}

    \begin{center}
    \adjustimage{max size={0.9\linewidth}{0.9\paperheight}}{Oving3-revidert-v1_files/Oving3-revidert-v1_54_1.png}
    \end{center}
    { \hspace*{\fill} \\}
    
    \begin{center}
    \adjustimage{max size={0.9\linewidth}{0.9\paperheight}}{Oving3-revidert-v1_files/Oving3-revidert-v1_54_2.png}
    \end{center}
    { \hspace*{\fill} \\}
    
    \begin{center}
    \adjustimage{max size={0.9\linewidth}{0.9\paperheight}}{Oving3-revidert-v1_files/Oving3-revidert-v1_54_3.png}
    \end{center}
    { \hspace*{\fill} \\}
    
    \hypertarget{forklaring}{%
\subsection*{Forklaring}\label{forklaring}}

I første grafen over kan en se at en har røttene i de tettbebygde
aksessnettene som har høyest ``Degree centrality''. Dette er fordi de
kommuniserer med ende-systemene som det ofte er flest av. Deres
robusthet er ikke særlig sterk da det ikke er noen særlig fail-safe om
de skulle feile, noe som kan kutte endesystemer fra nettet.

I andre grafen ser en den mest sentrale noden basert på Betweenness. At
det er nettop d0 er nok fordi den har en ekstra kant til core8 og får
derfor ekstra veier gjennom seg. Ellers kunne det vært en gitt rotnode
for regional-nettene ut ifra kjernen. For det er nettop disse alle
trafiikken må gjennom for å nå noder i regionalnettene. At disse nodene
står alene for al trafiiken ut til reional-nettene kan gjøre nettet
svært sårbart, og bør kanskje forsterkes.

I siste grafen er den mest sentrale noden basert på ``Closeness''. Her
er det svært robust at en har en kjerne-node som den noden med kortest
vei til alle. Dermed kan distribuere informasjon på en mest mulig
effektiv måte.

    \hypertarget{del-3-angrep-og-robusthet}{%
\section*{Del 3: Angrep og robusthet}\label{del-3-angrep-og-robusthet}}

Her skal vi analysere og diskutere det simulerte nettverket fra oppgave
2.3 i dybden. I denne oppgaven vil vi bruke en utvidet versjon av
nettverket. Kjør cellen for å generere nettverket

    \begin{tcolorbox}[breakable, size=fbox, boxrule=1pt, pad at break*=1mm,colback=cellbackground, colframe=cellborder]
\prompt{In}{incolor}{35}{\boxspacing}
\begin{Verbatim}[commandchars=\\\{\}]
\PY{n}{cg} \PY{o}{=} \PY{n}{ConstructedGraph}\PY{p}{(}\PY{n}{expanded}\PY{o}{=}\PY{k+kc}{True}\PY{p}{,} \PY{n}{seed}\PY{o}{=}\PY{n}{student\PYZus{}seed}\PY{p}{)}
\PY{n}{cg}\PY{o}{.}\PY{n}{draw}\PY{p}{(}\PY{p}{)}
\end{Verbatim}
\end{tcolorbox}

    \begin{center}
    \adjustimage{max size={0.9\linewidth}{0.9\paperheight}}{Oving3-revidert-v1_files/Oving3-revidert-v1_57_0.png}
    \end{center}
    { \hspace*{\fill} \\}
    
    \hypertarget{oppgave-3.1}{%
\subsection*{Oppgave 3.1}\label{oppgave-3.1}}

Bruk de numeriske verdiene for de forskjellige centrality-målene for å
finne hvilke noder i nettverket som er viktigst. Er det noen av nodene
som overrasker deg?

Her forventer vi å se at studenten bruker metoder som tidligere har
blitt introdusert, til å finne de viktigste nodene i grafen. Deretter
forventer vi å se en kort drøftende tekst på hvilke noder dette gjelder.

    \hypertarget{forklaring}{%
\paragraph{Forklaring}\label{forklaring}}

Som en ser i grafen under er det tydelig at alle kjerne-nodene er
sentrale for driften av nettverket, uansett hvilken centrality en ser
på. De er de mest tilgjengelige siden grafen har like egenskaper som en
tree-graph. Kjernen er også robust da de har høy Degree centrality
mellom seg, som gjør det lett å kunne videreformidle informasjon om det
skjer en feil en kjerne-node.

Det som overasker er hvor mye nettverket er avhengige av robuste noder
rett utenfor kjernen, altså noder som a0 og c0. Om disse feiler vil en
kutte av en gitt fjerdel av nettverket siden betweenessen avhenger av de
alene for hele regional-nettverket. Det samme gjelder også for nodene
som ba0 og bc0 for å kunne nå endesystemene.

    \begin{tcolorbox}[breakable, size=fbox, boxrule=1pt, pad at break*=1mm,colback=cellbackground, colframe=cellborder]
\prompt{In}{incolor}{161}{\boxspacing}
\begin{Verbatim}[commandchars=\\\{\}]
\PY{c+c1}{\PYZsh{}Kode og forkaring her}
\PY{n+nb}{print}\PY{p}{(}\PY{l+s+s2}{\PYZdq{}}\PY{l+s+s2}{1. Degree Centrality}\PY{l+s+s2}{\PYZdq{}}\PY{p}{)}
\PY{n}{cg}\PY{o}{.}\PY{n}{draw\PYZus{}degree\PYZus{}centrality}\PY{p}{(}\PY{p}{)}
\end{Verbatim}
\end{tcolorbox}

    \begin{Verbatim}[commandchars=\\\{\}]
1. Degree Centrality
    \end{Verbatim}

    \begin{center}
    \adjustimage{max size={0.9\linewidth}{0.9\paperheight}}{Oving3-revidert-v1_files/Oving3-revidert-v1_60_1.png}
    \end{center}
    { \hspace*{\fill} \\}
    
    \begin{tcolorbox}[breakable, size=fbox, boxrule=1pt, pad at break*=1mm,colback=cellbackground, colframe=cellborder]
\prompt{In}{incolor}{162}{\boxspacing}
\begin{Verbatim}[commandchars=\\\{\}]
\PY{n+nb}{print}\PY{p}{(}\PY{l+s+s2}{\PYZdq{}}\PY{l+s+s2}{2. Closeness Centrality}\PY{l+s+s2}{\PYZdq{}}\PY{p}{)}
\PY{n}{cg}\PY{o}{.}\PY{n}{draw\PYZus{}closeness\PYZus{}centrality}\PY{p}{(}\PY{p}{)}
\end{Verbatim}
\end{tcolorbox}

    \begin{Verbatim}[commandchars=\\\{\}]
2. Closeness Centrality
    \end{Verbatim}

    \begin{center}
    \adjustimage{max size={0.9\linewidth}{0.9\paperheight}}{Oving3-revidert-v1_files/Oving3-revidert-v1_61_1.png}
    \end{center}
    { \hspace*{\fill} \\}
    
    \begin{tcolorbox}[breakable, size=fbox, boxrule=1pt, pad at break*=1mm,colback=cellbackground, colframe=cellborder]
\prompt{In}{incolor}{163}{\boxspacing}
\begin{Verbatim}[commandchars=\\\{\}]
\PY{n+nb}{print}\PY{p}{(}\PY{l+s+s2}{\PYZdq{}}\PY{l+s+s2}{3. Betweenness Centrality}\PY{l+s+s2}{\PYZdq{}}\PY{p}{)}
\PY{n}{cg}\PY{o}{.}\PY{n}{draw\PYZus{}betweenness\PYZus{}centrality}\PY{p}{(}\PY{p}{)}
\end{Verbatim}
\end{tcolorbox}

    \begin{Verbatim}[commandchars=\\\{\}]
3. Betweenness Centrality
    \end{Verbatim}

    \begin{center}
    \adjustimage{max size={0.9\linewidth}{0.9\paperheight}}{Oving3-revidert-v1_files/Oving3-revidert-v1_62_1.png}
    \end{center}
    { \hspace*{\fill} \\}
    
    \hypertarget{oppgave-3.2}{%
\subsection*{Oppgave 3.2}\label{oppgave-3.2}}

Bruk metoden delete\_random\_nodes for å simulere tilfeldige feil som
kan skje i nettverket. * Fjern en node. Tegn så grafen * Fjern tre
noder. Tegn så grafen

Kommenter skaden av nettverket

Her forventer vi å se at studenten klarer å printe ut to forskjellige
grafer, en der en node er fjernet og en der tre noder er fjernet. Disse
nodene skal være fjernet ved å bruke delete\_random\_nodes. Forventer
også et kort analyse over nettverket, og drøftning av skaden disse
feilene påførte nettverket. Her kan man dra inn metoder tidligere brukt

    \begin{tcolorbox}[breakable, size=fbox, boxrule=1pt, pad at break*=1mm,colback=cellbackground, colframe=cellborder]
\prompt{In}{incolor}{164}{\boxspacing}
\begin{Verbatim}[commandchars=\\\{\}]
\PY{c+c1}{\PYZsh{}Skriv koden her}
\PY{n}{c} \PY{o}{=} \PY{n}{ConstructedGraph}\PY{p}{(}\PY{n}{expanded}\PY{o}{=}\PY{k+kc}{True}\PY{p}{,} \PY{n}{seed}\PY{o}{=}\PY{l+m+mi}{10027}\PY{p}{)}
\PY{n}{graf}\PY{p}{,} \PY{n}{graf1}\PY{p}{,} \PY{n}{graf2} \PY{o}{=} \PY{n}{c}\PY{p}{,} \PY{n}{c}\PY{p}{,} \PY{n}{c}\PY{p}{,}
\PY{n}{graf1} \PY{o}{=} \PY{n}{graf1}\PY{o}{.}\PY{n}{delete\PYZus{}random\PYZus{}nodes}\PY{p}{(}\PY{p}{)}\PY{p}{;}
\PY{n}{graf2} \PY{o}{=} \PY{n}{graf2}\PY{o}{.}\PY{n}{delete\PYZus{}random\PYZus{}nodes}\PY{p}{(}\PY{l+m+mi}{3}\PY{p}{)}\PY{p}{;}

\PY{n}{graf1}\PY{o}{.}\PY{n}{draw}\PY{p}{(}\PY{p}{)}
\end{Verbatim}
\end{tcolorbox}

    \begin{Verbatim}[commandchars=\\\{\}]
Removed node bc0 using random\_fault
Removed node bc0 using random\_fault
Removed node ba31 using random\_fault
Removed node ba51 using random\_fault
    \end{Verbatim}

    \begin{center}
    \adjustimage{max size={0.9\linewidth}{0.9\paperheight}}{Oving3-revidert-v1_files/Oving3-revidert-v1_64_1.png}
    \end{center}
    { \hspace*{\fill} \\}
    
    \begin{tcolorbox}[breakable, size=fbox, boxrule=1pt, pad at break*=1mm,colback=cellbackground, colframe=cellborder]
\prompt{In}{incolor}{165}{\boxspacing}
\begin{Verbatim}[commandchars=\\\{\}]
\PY{n}{graf2}\PY{o}{.}\PY{n}{draw}\PY{p}{(}\PY{p}{)}
\end{Verbatim}
\end{tcolorbox}

    \begin{center}
    \adjustimage{max size={0.9\linewidth}{0.9\paperheight}}{Oving3-revidert-v1_files/Oving3-revidert-v1_65_0.png}
    \end{center}
    { \hspace*{\fill} \\}
    
    \begin{tcolorbox}[breakable, size=fbox, boxrule=1pt, pad at break*=1mm,colback=cellbackground, colframe=cellborder]
\prompt{In}{incolor}{166}{\boxspacing}
\begin{Verbatim}[commandchars=\\\{\}]
\PY{n}{graf}\PY{o}{.}\PY{n}{mark\PYZus{}nodes}\PY{p}{(}\PY{p}{[}\PY{l+s+s2}{\PYZdq{}}\PY{l+s+s2}{bc0}\PY{l+s+s2}{\PYZdq{}}\PY{p}{,}\PY{l+s+s2}{\PYZdq{}}\PY{l+s+s2}{ba31}\PY{l+s+s2}{\PYZdq{}}\PY{p}{,}\PY{l+s+s2}{\PYZdq{}}\PY{l+s+s2}{ba51}\PY{l+s+s2}{\PYZdq{}}\PY{p}{]}\PY{p}{)}
\end{Verbatim}
\end{tcolorbox}

    \begin{center}
    \adjustimage{max size={0.9\linewidth}{0.9\paperheight}}{Oving3-revidert-v1_files/Oving3-revidert-v1_66_0.png}
    \end{center}
    { \hspace*{\fill} \\}
    
    \hypertarget{node-feil}{%
\subsubsection*{1 Node-feil}\label{node-feil}}

Når en node feiler er det bc0 som går bort. Dette er svært alvorlig for
dette nettverket, da et helt regional-nett er avhengig av betweennessen
gjennom denne. Som vist i tredje grafen fører feilen av bc0 til at cirka
en fjerdel av nettet faller bort.

\hypertarget{node--feil}{%
\subsubsection*{3 Node -feil}\label{node--feil}}

I det andre tilfellet der 3 noder feiler mister vi den samme som før,
samt to ende-noder: ba31 og ba51. At disse to faller bort er ikke særlig
alvorlig for selve nettverket, da ikke mye avhenger av deres tilkobling.
Men at bc0 faller bort er katastrofalt.

\textbf{nb}: i de to første grafene blir de avkuttede nodene kastet i
midten, fant ingen fiks på det

    \hypertarget{oppgave-3.3}{%
\subsection*{Oppgave 3.3}\label{oppgave-3.3}}

Prøv å angripe nettverket ved å bruke en kombinasjon av logisk tenkning
og verdiene fra oppgave 3.1. Oppgaven her er å gjøre så mye skade som
mulig ved å fjerne tre noder.

For hver node du velger å fjerne, begrunn hvorfor.

Her forventer vi at studenten har forstått oppgaven og velger å fjerne
tre noder som er svært sentrale. Forventer å se begrunnelse/tankegang
for hver node som fjernes.

    \hypertarget{angrep}{%
\paragraph{Angrep}\label{angrep}}

Planen for å gjøre mest skade blir å prøve å separere så mange subnett
fra kjernen som mulig, dette kan gjøres ved å gjenkjenne de nodene med
høy betweenness. Disse er ofte knyttepunkter, dermed kan en bruke disse
til å prøve å isolere subnett

    \begin{tcolorbox}[breakable, size=fbox, boxrule=1pt, pad at break*=1mm,colback=cellbackground, colframe=cellborder]
\prompt{In}{incolor}{170}{\boxspacing}
\begin{Verbatim}[commandchars=\\\{\}]
\PY{n}{attackGraph} \PY{o}{=} \PY{n}{ConstructedGraph}\PY{p}{(}\PY{n}{expanded}\PY{o}{=}\PY{k+kc}{True}\PY{p}{,} \PY{n}{seed}\PY{o}{=}\PY{l+m+mi}{10027}\PY{p}{)}\PY{p}{;}

\PY{n}{attackGraph}\PY{o}{.}\PY{n}{mark\PYZus{}nodes}\PY{p}{(}\PY{p}{[}\PY{l+s+s2}{\PYZdq{}}\PY{l+s+s2}{a0}\PY{l+s+s2}{\PYZdq{}}\PY{p}{,}\PY{l+s+s2}{\PYZdq{}}\PY{l+s+s2}{d0}\PY{l+s+s2}{\PYZdq{}}\PY{p}{,} \PY{l+s+s2}{\PYZdq{}}\PY{l+s+s2}{core4}\PY{l+s+s2}{\PYZdq{}}\PY{p}{]}\PY{p}{)}
\end{Verbatim}
\end{tcolorbox}

    \begin{center}
    \adjustimage{max size={0.9\linewidth}{0.9\paperheight}}{Oving3-revidert-v1_files/Oving3-revidert-v1_70_0.png}
    \end{center}
    { \hspace*{\fill} \\}
    
    Det første vi gjør er å angripe nodene a0 og d0. Disse er to rotnoder
for nettverk utifra kjernen, og de er på hver sin side av nettverket,
som isolerer etterfølgerne deres, samt gjør det mulig for å videre
isolere subnett.

Deretter angriper vi core4, som dermed kutter ut core1 og rotnoden b0,
som kutter hele sitt tre ut av nettverket Dermed står en kun igjen med
fire separate subnett der kun to av de har kjerne-noder i seg, noe som
vil være katastrofalt

    \begin{tcolorbox}[breakable, size=fbox, boxrule=1pt, pad at break*=1mm,colback=cellbackground, colframe=cellborder]
\prompt{In}{incolor}{171}{\boxspacing}
\begin{Verbatim}[commandchars=\\\{\}]
\PY{n}{attackGraph}\PY{o}{.}\PY{n}{remove\PYZus{}node}\PY{p}{(}\PY{l+s+s2}{\PYZdq{}}\PY{l+s+s2}{a0}\PY{l+s+s2}{\PYZdq{}}\PY{p}{)}
\PY{n}{attackGraph}\PY{o}{.}\PY{n}{remove\PYZus{}node}\PY{p}{(}\PY{l+s+s2}{\PYZdq{}}\PY{l+s+s2}{d0}\PY{l+s+s2}{\PYZdq{}}\PY{p}{)}
\PY{n}{attackGraph}\PY{o}{.}\PY{n}{remove\PYZus{}node}\PY{p}{(}\PY{l+s+s2}{\PYZdq{}}\PY{l+s+s2}{core4}\PY{l+s+s2}{\PYZdq{}}\PY{p}{)}

\PY{n}{attackGraph}\PY{o}{.}\PY{n}{draw}\PY{p}{(}\PY{p}{)}
\end{Verbatim}
\end{tcolorbox}

    \begin{center}
    \adjustimage{max size={0.9\linewidth}{0.9\paperheight}}{Oving3-revidert-v1_files/Oving3-revidert-v1_72_0.png}
    \end{center}
    { \hspace*{\fill} \\}
    
    \hypertarget{oppgave-3.4}{%
\subsection*{Oppgave 3.4}\label{oppgave-3.4}}

En måte å se hvor mye skade som har blitt gjort på et nettverk er å
bruke noder i største partisjon, eller se på node degree ved hjelp av et
histogram.

Bruk disse metodene på oppgave 3.2 og 3.3, og diskuter skadene ut ifra
resultatene du nå har fått.

Her forventer vi å se at studenten klarer å vise et histogram og klarer
å bruke metoden for å finne noder i største partisjon. Vi forventer også
å se en drøftning av konsekvensene i de to foregående oppgavene, og at
studenten skjønner alvorligheten i målrettede angrep.

    \begin{tcolorbox}[breakable, size=fbox, boxrule=1pt, pad at break*=1mm,colback=cellbackground, colframe=cellborder]
\prompt{In}{incolor}{172}{\boxspacing}
\begin{Verbatim}[commandchars=\\\{\}]
\PY{c+c1}{\PYZsh{}Kode her}
\PY{n}{main} \PY{o}{=} \PY{n}{ConstructedGraph}\PY{p}{(}\PY{n}{expanded}\PY{o}{=}\PY{k+kc}{True}\PY{p}{,} \PY{n}{seed}\PY{o}{=}\PY{l+m+mi}{10027}\PY{p}{)}\PY{p}{;}
\PY{n+nb}{print}\PY{p}{(}\PY{l+s+s2}{\PYZdq{}}\PY{l+s+s2}{Utgangspunktet}\PY{l+s+s2}{\PYZdq{}}\PY{p}{)}
\PY{n}{cg}\PY{o}{.}\PY{n}{histogram}\PY{p}{(}\PY{p}{)}
\PY{n+nb}{print}\PY{p}{(}\PY{l+s+s2}{\PYZdq{}}\PY{l+s+s2}{Største partisjon:}\PY{l+s+s2}{\PYZdq{}}\PY{p}{,} \PY{n}{main}\PY{o}{.}\PY{n}{get\PYZus{}largest\PYZus{}components\PYZus{}size}\PY{p}{(}\PY{p}{)}\PY{p}{)}

\PY{n+nb}{print}\PY{p}{(}\PY{l+s+s2}{\PYZdq{}}\PY{l+s+s2}{Oppgave 3.2}\PY{l+s+s2}{\PYZdq{}}\PY{p}{)}
\PY{n}{graf2}\PY{o}{.}\PY{n}{histogram}\PY{p}{(}\PY{p}{)}\PY{p}{;}
\PY{n+nb}{print}\PY{p}{(}\PY{l+s+s2}{\PYZdq{}}\PY{l+s+s2}{Største partisjon:}\PY{l+s+s2}{\PYZdq{}}\PY{p}{,} \PY{n}{graf2}\PY{o}{.}\PY{n}{get\PYZus{}largest\PYZus{}components\PYZus{}size}\PY{p}{(}\PY{p}{)}\PY{p}{)}

\PY{n+nb}{print}\PY{p}{(}\PY{l+s+s2}{\PYZdq{}}\PY{l+s+s2}{Oppgave 3.3}\PY{l+s+s2}{\PYZdq{}}\PY{p}{)}
\PY{n}{attackGraph}\PY{o}{.}\PY{n}{histogram}\PY{p}{(}\PY{p}{)}\PY{p}{;}
\PY{n+nb}{print}\PY{p}{(}\PY{l+s+s2}{\PYZdq{}}\PY{l+s+s2}{Største partisjon:}\PY{l+s+s2}{\PYZdq{}}\PY{p}{,} \PY{n}{attackGraph}\PY{o}{.}\PY{n}{get\PYZus{}largest\PYZus{}components\PYZus{}size}\PY{p}{(}\PY{p}{)}\PY{p}{)}
\end{Verbatim}
\end{tcolorbox}

    \begin{Verbatim}[commandchars=\\\{\}]
Utgangspunktet
    \end{Verbatim}

    \begin{center}
    \adjustimage{max size={0.9\linewidth}{0.9\paperheight}}{Oving3-revidert-v1_files/Oving3-revidert-v1_74_1.png}
    \end{center}
    { \hspace*{\fill} \\}
    
    \begin{Verbatim}[commandchars=\\\{\}]
Største partisjon: 129
Oppgave 3.2
    \end{Verbatim}

    \begin{center}
    \adjustimage{max size={0.9\linewidth}{0.9\paperheight}}{Oving3-revidert-v1_files/Oving3-revidert-v1_74_3.png}
    \end{center}
    { \hspace*{\fill} \\}
    
    \begin{Verbatim}[commandchars=\\\{\}]
Største partisjon: 114
Oppgave 3.3
    \end{Verbatim}

    \begin{center}
    \adjustimage{max size={0.9\linewidth}{0.9\paperheight}}{Oving3-revidert-v1_files/Oving3-revidert-v1_74_5.png}
    \end{center}
    { \hspace*{\fill} \\}
    
    \begin{Verbatim}[commandchars=\\\{\}]
Største partisjon: 34
    \end{Verbatim}

    \hypertarget{forklaring}{%
\section*{Forklaring}\label{forklaring}}

\hypertarget{section*}{%
\subsection*{3.2}\label{section*}}

Her kan en se at feilene ikke hadde betydelig utslag på histografen. Den
største partisjonen gikk også fra 129 til 114, noe som ikke nødendgivs
er katastrofalt for nettvkerket, men samtidig svært alvorlig.

\hypertarget{section*-1}{%
\subsection*{3.3}\label{section*-1}}

Utslaget ved dette angrepet vises best i at noder som tidligere haddde 2
eller 4 kanter, nå har 1 eller 3. Dette reflekterer vårt mål om å dele
nettverket inn i subnett. Største partisjonen ble også redusert til 34!
Noe som er katastrofalt for funksjonaliteten til nettverket. Dette viser
at planlagt angrep av ondsinnde aktører vil kunne føre til ekstreme
konsekvenser.

    \hypertarget{oppgave-3.5}{%
\subsection*{Oppgave 3.5}\label{oppgave-3.5}}

Forklar hva som menes med begrepet noder i største partisjon. Hva kan
være fordeler og ulemper ved å kun bruke noder i største partisjon og
histogram som pålitelighetsmål? Er dette fornuftig i vårt tilfelle?

Her forventer vi å se at studenten har forstått begrepet noder i største
partisjon, og hvorfor dette kan brukes til å vise sårbarhet. Vi
forventer også en drøftning av hvorfor/hvorfor ikke, det kan være lurt å
bruke noder i største partisjon og histogram som de eneste
pålitelighetsmålene.

    \hypertarget{forklaring}{%
\section*{Forklaring}\label{forklaring}}

\hypertarget{noder-i-stuxf8rst-partisjon}{%
\subsubsection*{Noder i størst
partisjon}\label{noder-i-stuxf8rst-partisjon}}

Begrepet betyr at en ser på hvor mange noder den største partisjonen i
grafen inneholder før og etter en endring. I vårt tilfelle så vi først
på det totale antall noder som er sammnenhengende i grafen, og så etter
et angrep telte vi hvor mange noder vi hadde ``mistet''. Å bruke dette
som en målestokk avhenger av hva slags nettverk en bruker. I et
Peer-To-Peer nettverk vil det være et ekstremt nyttig verktøy da en ser
på nodene som ``likeverdige''. Men i andre sammenhnger da en har noder
som er dedikerte til funksjonalitet som leverandører av en tjeneste(som
for eksmpel servere), og noen som kun er forbrukere, vil en først og
fremst fokusere på de få serverene og ikke nødvendigvis de mange
forbrukerne. Dermed vil ikke hvor mange noder man mistet være fokuset,
men hvilke spesielle noder man mistet.

\hypertarget{histogram}{%
\subsubsection*{Histogram}\label{histogram}}

I nettverk med mange Stjerne-liknende grafer i seg kan histogram ganske
effektivt sjekke for om man har mistet spesielle typer noder, for
eksempel har serevere ett til mange relasjoner med noder i nettverket,
da kan man sjekke for om man har mistet noder med nettop mange kanter i
seg. Hvis man da krysjekker med metoden over vil man kunne finne mye
informasjon om hvilke, og hvilke typer noder man har mistet.

Disse to metodene kan flettes godt med hverandre for å drøfte
sårbarheter i nettverk, det er selvsagt ikke lurt å kun bruke disse
sammen, og ihvertfall ikke hver for seg. Men de er fine for å få et
ovrblikk.

    \hypertarget{oppgave-3.6}{%
\subsection*{Oppgave 3.6}\label{oppgave-3.6}}

Under ser du metoden for å lage en graf som sammenligner angrep med
forskjellige metoder. Bruk metoden på det originale nettverket og
sammenlign med det samme nettverket der du har lagt inn tre ekstra
kanter som redundans. Diskuter kort effekten av ekstra redundans.

Her forventer vi å se at studenten har klart å bruke metoden som er gitt
til å vise skaden på det originale nettverket. Deretter forventer vi å
se at studenten klarer å legge inn tre ekstra kanter, og begrunne
hvorfor disse tre kantene er valgt. Til slutt vil vi se en kort
diskusjon av effekten vi får av å legge til ekstra kanter i et nettverk.

    \begin{tcolorbox}[breakable, size=fbox, boxrule=1pt, pad at break*=1mm,colback=cellbackground, colframe=cellborder]
\prompt{In}{incolor}{173}{\boxspacing}
\begin{Verbatim}[commandchars=\\\{\}]
\PY{k}{def} \PY{n+nf}{get\PYZus{}attack\PYZus{}graph}\PY{p}{(}\PY{n}{G}\PY{p}{)}\PY{p}{:}
    \PY{n}{count} \PY{o}{=} \PY{n}{G}\PY{o}{.}\PY{n}{number\PYZus{}of\PYZus{}nodes}\PY{p}{(}\PY{p}{)}
    \PY{n}{outputs} \PY{o}{=} \PY{p}{[}\PY{p}{[}\PY{l+m+mi}{0} \PY{k}{for} \PY{n}{\PYZus{}} \PY{o+ow}{in} \PY{n+nb}{range}\PY{p}{(}\PY{n}{count}\PY{p}{)}\PY{p}{]} \PY{k}{for} \PY{n}{\PYZus{}} \PY{o+ow}{in} \PY{n+nb}{range}\PY{p}{(}\PY{l+m+mi}{4}\PY{p}{)}\PY{p}{]}
    \PY{n}{graphs} \PY{o}{=} \PY{p}{[}\PY{n}{G} \PY{k}{for} \PY{n}{\PYZus{}} \PY{o+ow}{in} \PY{n+nb}{range}\PY{p}{(}\PY{l+m+mi}{4}\PY{p}{)}\PY{p}{]}
    \PY{n}{x} \PY{o}{=} \PY{n+nb}{range}\PY{p}{(}\PY{l+m+mi}{0}\PY{p}{,}\PY{n}{count}\PY{p}{)}

    \PY{k}{for} \PY{n}{i} \PY{o+ow}{in} \PY{n}{x}\PY{p}{:}
        \PY{k}{for} \PY{n}{j}\PY{p}{,} \PY{n}{graph} \PY{o+ow}{in} \PY{n+nb}{enumerate}\PY{p}{(}\PY{n}{graphs}\PY{p}{)}\PY{p}{:}
            \PY{n}{outputs}\PY{p}{[}\PY{n}{j}\PY{p}{]}\PY{p}{[}\PY{n}{i}\PY{p}{]} \PY{o}{=} \PY{n}{graphs}\PY{p}{[}\PY{n}{j}\PY{p}{]}\PY{o}{.}\PY{n}{get\PYZus{}largest\PYZus{}components\PYZus{}size}\PY{p}{(}\PY{p}{)}
        \PY{n}{graphs}\PY{p}{[}\PY{l+m+mi}{0}\PY{p}{]} \PY{o}{=} \PY{n}{graphs}\PY{p}{[}\PY{l+m+mi}{0}\PY{p}{]}\PY{o}{.}\PY{n}{delete\PYZus{}random\PYZus{}nodes}\PY{p}{(}\PY{n}{print\PYZus{}result}\PY{o}{=}\PY{k+kc}{False}\PY{p}{)}
        \PY{n}{graphs}\PY{p}{[}\PY{l+m+mi}{1}\PY{p}{]} \PY{o}{=} \PY{n}{graphs}\PY{p}{[}\PY{l+m+mi}{1}\PY{p}{]}\PY{o}{.}\PY{n}{delete\PYZus{}nodes\PYZus{}attack}\PY{p}{(}\PY{n}{centrality\PYZus{}index}\PY{o}{=}\PY{l+s+s2}{\PYZdq{}}\PY{l+s+s2}{degree}\PY{l+s+s2}{\PYZdq{}}\PY{p}{,}\PY{n}{print\PYZus{}result}\PY{o}{=}\PY{k+kc}{False}\PY{p}{)}
        \PY{n}{graphs}\PY{p}{[}\PY{l+m+mi}{2}\PY{p}{]} \PY{o}{=} \PY{n}{graphs}\PY{p}{[}\PY{l+m+mi}{2}\PY{p}{]}\PY{o}{.}\PY{n}{delete\PYZus{}nodes\PYZus{}attack}\PY{p}{(}\PY{n}{centrality\PYZus{}index}\PY{o}{=}\PY{l+s+s2}{\PYZdq{}}\PY{l+s+s2}{closeness}\PY{l+s+s2}{\PYZdq{}}\PY{p}{,}\PY{n}{print\PYZus{}result}\PY{o}{=}\PY{k+kc}{False}\PY{p}{)}
        \PY{n}{graphs}\PY{p}{[}\PY{l+m+mi}{3}\PY{p}{]} \PY{o}{=} \PY{n}{graphs}\PY{p}{[}\PY{l+m+mi}{3}\PY{p}{]}\PY{o}{.}\PY{n}{delete\PYZus{}nodes\PYZus{}attack}\PY{p}{(}\PY{n}{centrality\PYZus{}index}\PY{o}{=}\PY{l+s+s2}{\PYZdq{}}\PY{l+s+s2}{betweenness}\PY{l+s+s2}{\PYZdq{}}\PY{p}{,}\PY{n}{print\PYZus{}result}\PY{o}{=}\PY{k+kc}{False}\PY{p}{)}

    \PY{n}{bc1}\PY{p}{,}\PY{n}{bc2}\PY{p}{,}\PY{n}{bc3}\PY{p}{,}\PY{n}{bc4} \PY{o}{=} \PY{n}{outputs}
    \PY{n}{plt}\PY{o}{.}\PY{n}{plot}\PY{p}{(}\PY{n}{x}\PY{p}{,}\PY{n}{bc1}\PY{p}{,}\PY{n}{color}\PY{o}{=}\PY{l+s+s2}{\PYZdq{}}\PY{l+s+s2}{red}\PY{l+s+s2}{\PYZdq{}}\PY{p}{,}\PY{n}{label}\PY{o}{=}\PY{l+s+s2}{\PYZdq{}}\PY{l+s+s2}{Random faults}\PY{l+s+s2}{\PYZdq{}}\PY{p}{)}
    \PY{n}{plt}\PY{o}{.}\PY{n}{plot}\PY{p}{(}\PY{n}{x}\PY{p}{,}\PY{n}{bc2}\PY{p}{,}\PY{n}{color}\PY{o}{=}\PY{l+s+s2}{\PYZdq{}}\PY{l+s+s2}{green}\PY{l+s+s2}{\PYZdq{}}\PY{p}{,}\PY{n}{label}\PY{o}{=}\PY{l+s+s2}{\PYZdq{}}\PY{l+s+s2}{Degree centrality}\PY{l+s+s2}{\PYZdq{}}\PY{p}{)}
    \PY{n}{plt}\PY{o}{.}\PY{n}{plot}\PY{p}{(}\PY{n}{x}\PY{p}{,}\PY{n}{bc3}\PY{p}{,}\PY{n}{color}\PY{o}{=}\PY{l+s+s2}{\PYZdq{}}\PY{l+s+s2}{blue}\PY{l+s+s2}{\PYZdq{}}\PY{p}{,}\PY{n}{label}\PY{o}{=}\PY{l+s+s2}{\PYZdq{}}\PY{l+s+s2}{Closeness Centrality}\PY{l+s+s2}{\PYZdq{}}\PY{p}{)}
    \PY{n}{plt}\PY{o}{.}\PY{n}{plot}\PY{p}{(}\PY{n}{x}\PY{p}{,}\PY{n}{bc4}\PY{p}{,}\PY{n}{color}\PY{o}{=}\PY{l+s+s2}{\PYZdq{}}\PY{l+s+s2}{orange}\PY{l+s+s2}{\PYZdq{}}\PY{p}{,} \PY{n}{label}\PY{o}{=}\PY{l+s+s2}{\PYZdq{}}\PY{l+s+s2}{Betweenness Centrality}\PY{l+s+s2}{\PYZdq{}}\PY{p}{)}
    \PY{n}{plt}\PY{o}{.}\PY{n}{ylabel}\PY{p}{(}\PY{l+s+s2}{\PYZdq{}}\PY{l+s+s2}{Amount of nodes in largest partition}\PY{l+s+s2}{\PYZdq{}}\PY{p}{)}
    \PY{n}{plt}\PY{o}{.}\PY{n}{xlabel}\PY{p}{(}\PY{l+s+s2}{\PYZdq{}}\PY{l+s+s2}{Amount of nodes in removed}\PY{l+s+s2}{\PYZdq{}}\PY{p}{)}
    \PY{n}{plt}\PY{o}{.}\PY{n}{legend}\PY{p}{(}\PY{n}{loc}\PY{o}{=}\PY{l+s+s2}{\PYZdq{}}\PY{l+s+s2}{upper right}\PY{l+s+s2}{\PYZdq{}}\PY{p}{)}
    \PY{n}{plt}\PY{o}{.}\PY{n}{show}\PY{p}{(}\PY{p}{)}
\end{Verbatim}
\end{tcolorbox}

    \begin{tcolorbox}[breakable, size=fbox, boxrule=1pt, pad at break*=1mm,colback=cellbackground, colframe=cellborder]
\prompt{In}{incolor}{174}{\boxspacing}
\begin{Verbatim}[commandchars=\\\{\}]
\PY{n}{org} \PY{o}{=} \PY{n}{ConstructedGraph}\PY{p}{(}\PY{n}{expanded}\PY{o}{=}\PY{k+kc}{True}\PY{p}{,} \PY{n}{seed}\PY{o}{=}\PY{l+m+mi}{10027}\PY{p}{)}

\PY{n}{get\PYZus{}attack\PYZus{}graph}\PY{p}{(}\PY{n}{org}\PY{p}{)}


\PY{n}{org}\PY{o}{.}\PY{n}{add\PYZus{}node}\PY{p}{(}\PY{l+s+s2}{\PYZdq{}}\PY{l+s+s2}{r0}\PY{l+s+s2}{\PYZdq{}}\PY{p}{)}
\PY{n}{org}\PY{o}{.}\PY{n}{add\PYZus{}node}\PY{p}{(}\PY{l+s+s2}{\PYZdq{}}\PY{l+s+s2}{r1}\PY{l+s+s2}{\PYZdq{}}\PY{p}{)}
\PY{n}{org}\PY{o}{.}\PY{n}{add\PYZus{}node}\PY{p}{(}\PY{l+s+s2}{\PYZdq{}}\PY{l+s+s2}{r2}\PY{l+s+s2}{\PYZdq{}}\PY{p}{)}

\PY{n}{org}\PY{o}{.}\PY{n}{add\PYZus{}edges\PYZus{}from}\PY{p}{(}\PY{p}{[}
    \PY{p}{(}\PY{l+s+s2}{\PYZdq{}}\PY{l+s+s2}{r0}\PY{l+s+s2}{\PYZdq{}}\PY{p}{,}\PY{l+s+s2}{\PYZdq{}}\PY{l+s+s2}{a4}\PY{l+s+s2}{\PYZdq{}}\PY{p}{)}\PY{p}{,}
    \PY{p}{(}\PY{l+s+s2}{\PYZdq{}}\PY{l+s+s2}{r0}\PY{l+s+s2}{\PYZdq{}}\PY{p}{,}\PY{l+s+s2}{\PYZdq{}}\PY{l+s+s2}{core1}\PY{l+s+s2}{\PYZdq{}}\PY{p}{)}\PY{p}{,}
    \PY{p}{(}\PY{l+s+s2}{\PYZdq{}}\PY{l+s+s2}{r0}\PY{l+s+s2}{\PYZdq{}}\PY{p}{,}\PY{l+s+s2}{\PYZdq{}}\PY{l+s+s2}{b1}\PY{l+s+s2}{\PYZdq{}}\PY{p}{)}\PY{p}{,}
    \PY{p}{(}\PY{l+s+s2}{\PYZdq{}}\PY{l+s+s2}{r1}\PY{l+s+s2}{\PYZdq{}}\PY{p}{,}\PY{l+s+s2}{\PYZdq{}}\PY{l+s+s2}{b4}\PY{l+s+s2}{\PYZdq{}}\PY{p}{)}\PY{p}{,}
    \PY{p}{(}\PY{l+s+s2}{\PYZdq{}}\PY{l+s+s2}{r1}\PY{l+s+s2}{\PYZdq{}}\PY{p}{,}\PY{l+s+s2}{\PYZdq{}}\PY{l+s+s2}{d1}\PY{l+s+s2}{\PYZdq{}}\PY{p}{)}\PY{p}{,}
    \PY{p}{(}\PY{l+s+s2}{\PYZdq{}}\PY{l+s+s2}{r2}\PY{l+s+s2}{\PYZdq{}}\PY{p}{,}\PY{l+s+s2}{\PYZdq{}}\PY{l+s+s2}{c1}\PY{l+s+s2}{\PYZdq{}}\PY{p}{)}\PY{p}{,}
    \PY{p}{(}\PY{l+s+s2}{\PYZdq{}}\PY{l+s+s2}{r2}\PY{l+s+s2}{\PYZdq{}}\PY{p}{,}\PY{l+s+s2}{\PYZdq{}}\PY{l+s+s2}{d4}\PY{l+s+s2}{\PYZdq{}}\PY{p}{)}\PY{p}{,}
    \PY{p}{(}\PY{l+s+s2}{\PYZdq{}}\PY{l+s+s2}{r2}\PY{l+s+s2}{\PYZdq{}}\PY{p}{,}\PY{l+s+s2}{\PYZdq{}}\PY{l+s+s2}{core7}\PY{l+s+s2}{\PYZdq{}}\PY{p}{)}
\PY{p}{]}\PY{p}{)}
\end{Verbatim}
\end{tcolorbox}

    \begin{center}
    \adjustimage{max size={0.9\linewidth}{0.9\paperheight}}{Oving3-revidert-v1_files/Oving3-revidert-v1_80_0.png}
    \end{center}
    { \hspace*{\fill} \\}
    
    \hypertarget{strategi}{%
\subsubsection*{Strategi}\label{strategi}}

Etter angrepet er det tydelig at nodene rett utenfor kjernen er det mest
utsatte, dermed virker det naturlig å fortsterke deres koblinger til
resten av nettverket. Å utvide kjernen sine koblinger mellom grenene, og
dermed øke antall stier til hver ende-node er beste måten å styrke
grafen.

    \begin{tcolorbox}[breakable, size=fbox, boxrule=1pt, pad at break*=1mm,colback=cellbackground, colframe=cellborder]
\prompt{In}{incolor}{175}{\boxspacing}
\begin{Verbatim}[commandchars=\\\{\}]
\PY{n}{get\PYZus{}attack\PYZus{}graph}\PY{p}{(}\PY{n}{org}\PY{p}{)}
\end{Verbatim}
\end{tcolorbox}

    \begin{center}
    \adjustimage{max size={0.9\linewidth}{0.9\paperheight}}{Oving3-revidert-v1_files/Oving3-revidert-v1_82_0.png}
    \end{center}
    { \hspace*{\fill} \\}
    
    \hypertarget{resultat}{%
\subsubsection*{Resultat}\label{resultat}}

Etter de tre forsterkende nodene er betydelig bedre. Man en kan se at
grafen ikke er like bratt. Naturligvis vil det å fjerne flere enn 15
noder gjøre stor skade, uansett hvor mange kanter man legger til som
redundans. Effekten av å legge til kanter vil som regel gjøre nettverket
mer robust, i form av at det tåler mer trafikk og håndterer at noder
feiler bedre. Det vil også gjøre det mer effektivt ved at man får flere
stier. Det som er ulempen er at det blir mer kostbart å opprettholde og
vedlikeholde.

    \hypertarget{del-4-sikring-av-vdes}{%
\section*{Del 4: Sikring av VDES}\label{del-4-sikring-av-vdes}}

Du er nå ansatt som en sikkerhetsingeniør i VDES; og har som oppgave å
sikrenettverket mot angrep, både mot ondsinnede handlinger og tilfeldige
feil . Her er det viktig å analysere nettverket, finne low-hanging
fruits og legge på ekstra redundans der det trengs, uten at kostnadene
skal bli unødvendig høye. Bruk metodene som er lært i øvingen for å
analysere og sikre nettverket. Forklar dine valg. Drøft konsekvensene av
at ondsinnede aktører får tak i informasjonen rundt nettverket ditt.

Analysen skal vise en gjennomgående forståelse av grafteori og
grafteoretisk strukturell analyse. Bruk av fagbegreper og relevante
begreper blir vektlagt. Oppgaven skal være kort og konsis og maks 800
ord. Start oppgaven med et bilde av nettverket og vis til figurer,
histogrammer og grafer hvordan du vil sikre nettverket. Utforsk gjerne
med å legge til redundans, teste angrep og vurdere tiltakene med
histogram over node degree. Inkluder robusthetsgrafen og bruk den for å
se på effekten. Bruk koden og resultatene som dokumentasjon for
tiltakene og hva din anbefaling blir til VDES selskapet.

Netverket du skal analysere finner du vek å kjøre kodeblokken under:

    \begin{tcolorbox}[breakable, size=fbox, boxrule=1pt, pad at break*=1mm,colback=cellbackground, colframe=cellborder]
\prompt{In}{incolor}{177}{\boxspacing}
\begin{Verbatim}[commandchars=\\\{\}]
\PY{n}{r}\PY{o}{.}\PY{n}{seed}\PY{p}{(}\PY{n}{student\PYZus{}seed}\PY{p}{)}
\PY{n}{boat\PYZus{}count} \PY{o}{=} \PY{n}{r}\PY{o}{.}\PY{n}{randint}\PY{p}{(}\PY{l+m+mi}{4}\PY{p}{,}\PY{l+m+mi}{9}\PY{p}{)}
\PY{n}{satellite\PYZus{}count} \PY{o}{=} \PY{n}{r}\PY{o}{.}\PY{n}{randint}\PY{p}{(}\PY{l+m+mi}{1}\PY{p}{,}\PY{l+m+mi}{3}\PY{p}{)}
\PY{n}{radio\PYZus{}tower\PYZus{}count} \PY{o}{=} \PY{n}{r}\PY{o}{.}\PY{n}{randint}\PY{p}{(}\PY{l+m+mi}{2}\PY{p}{,}\PY{l+m+mi}{5}\PY{p}{)}

\PY{n}{vdesGraph} \PY{o}{=} \PY{n}{VDESGraph}\PY{p}{(}\PY{n}{boat\PYZus{}count}\PY{p}{,}\PY{n}{satellite\PYZus{}count}\PY{p}{,}\PY{n}{radio\PYZus{}tower\PYZus{}count}\PY{p}{)}
\PY{n}{vdesGraph}\PY{o}{.}\PY{n}{draw}\PY{p}{(}\PY{p}{)}
\PY{n+nb}{print}\PY{p}{(}\PY{l+s+s2}{\PYZdq{}}\PY{l+s+s2}{Histogram av grafen}\PY{l+s+s2}{\PYZdq{}}\PY{p}{)}
\PY{n}{vdesGraph}\PY{o}{.}\PY{n}{histogram}\PY{p}{(}\PY{p}{)}\PY{p}{;}
\end{Verbatim}
\end{tcolorbox}

    \begin{center}
    \adjustimage{max size={0.9\linewidth}{0.9\paperheight}}{Oving3-revidert-v1_files/Oving3-revidert-v1_85_0.png}
    \end{center}
    { \hspace*{\fill} \\}
    
    \begin{Verbatim}[commandchars=\\\{\}]
Histogram av grafen
    \end{Verbatim}

    \begin{center}
    \adjustimage{max size={0.9\linewidth}{0.9\paperheight}}{Oving3-revidert-v1_files/Oving3-revidert-v1_85_2.png}
    \end{center}
    { \hspace*{\fill} \\}
    
    \hypertarget{styrker}{%
\subsubsection*{Styrker}\label{styrker}}

Gjennom histogrammet kan en se at det er kun to noder som har en kant i
seg. I grafen ser en også at disse to kun er radiotårn, noe som er
ideelt da de er de billigste elementene av nettverket, og at det ikke er
risiko for tap av menneskeliv om de går tapt. Det viktigste er nemlig at
båtene minst har to noder å kommunisere med, slik at en kan oprettholde
kommunkasjon med omverdenen. Siden vårt nettverk har det, er vi sikret
en type ``fail-safe'' ved at enhver båt sannysnligvis alltid har minst
en node å kommunisere med.

\hypertarget{svakheter}{%
\subsubsection*{Svakheter}\label{svakheter}}

Et såpass bra knyttet nettverk kan også være en svakhet, fordi om en
ondsinnet aktør finner en såkalt ``zero day vulnerability'' og gjennom
den, eller noen fysiske svakheter få injisert nettverket med malware,
vil alle noder være utsatt for angrepet. Siden det er mange noder med
\textgreater3 grader vil en dårlig pakke også kunne spre seg fort over
nettverket.

\hypertarget{tiltak}{%
\subsubsection*{Tiltak}\label{tiltak}}

Siden satelitt-dekningen her er såpass god kan en eliminere risikoen ved
å ha utsatte radiotårn på bakken. Ved metoden reduce kan en redusere
kommunikasjonen med tårnene og heller bruke de som en failsafe om
satelittene skulle feile. Eller så kan en lage en type rettet graf den
en kun tillater enveis kommunkasjon mot land, om en er i nød eller
liknende, og heller tillate toveis-kommunikasjon som en failsafe.

    \begin{tcolorbox}[breakable, size=fbox, boxrule=1pt, pad at break*=1mm,colback=cellbackground, colframe=cellborder]
\prompt{In}{incolor}{178}{\boxspacing}
\begin{Verbatim}[commandchars=\\\{\}]
\PY{n}{vdes} \PY{o}{=} \PY{n}{VDESGraph}\PY{p}{(}\PY{n}{boat\PYZus{}count}\PY{p}{,}\PY{n}{satellite\PYZus{}count}\PY{p}{,}\PY{n}{radio\PYZus{}tower\PYZus{}count}\PY{p}{)}

\PY{n+nb}{print}\PY{p}{(}\PY{l+s+s2}{\PYZdq{}}\PY{l+s+s2}{Nettverk uten radiotårn}\PY{l+s+s2}{\PYZdq{}}\PY{p}{)}
\PY{n}{vdes}\PY{o}{.}\PY{n}{remove\PYZus{}node}\PY{p}{(}\PY{l+s+s2}{\PYZdq{}}\PY{l+s+s2}{radio\PYZus{}tower0}\PY{l+s+s2}{\PYZdq{}}\PY{p}{)}
\PY{n}{vdes}\PY{o}{.}\PY{n}{remove\PYZus{}node}\PY{p}{(}\PY{l+s+s2}{\PYZdq{}}\PY{l+s+s2}{radio\PYZus{}tower1}\PY{l+s+s2}{\PYZdq{}}\PY{p}{)}
\PY{n}{vdes}\PY{o}{.}\PY{n}{remove\PYZus{}node}\PY{p}{(}\PY{l+s+s2}{\PYZdq{}}\PY{l+s+s2}{radio\PYZus{}tower2}\PY{l+s+s2}{\PYZdq{}}\PY{p}{)}
\PY{n}{vdes}\PY{o}{.}\PY{n}{remove\PYZus{}node}\PY{p}{(}\PY{l+s+s2}{\PYZdq{}}\PY{l+s+s2}{radio\PYZus{}tower3}\PY{l+s+s2}{\PYZdq{}}\PY{p}{)}
\PY{n}{vdes}\PY{o}{.}\PY{n}{remove\PYZus{}node}\PY{p}{(}\PY{l+s+s2}{\PYZdq{}}\PY{l+s+s2}{radio\PYZus{}tower4}\PY{l+s+s2}{\PYZdq{}}\PY{p}{)}
\PY{n}{vdes}\PY{o}{.}\PY{n}{draw}\PY{p}{(}\PY{p}{)}
\PY{n+nb}{print}\PY{p}{(}\PY{l+s+s2}{\PYZdq{}}\PY{l+s+s2}{Histogram uten radiotårn}\PY{l+s+s2}{\PYZdq{}}\PY{p}{)}
\PY{n}{vdes}\PY{o}{.}\PY{n}{histogram}\PY{p}{(}\PY{p}{)}\PY{p}{;}
\end{Verbatim}
\end{tcolorbox}

    \begin{Verbatim}[commandchars=\\\{\}]
Nettverk uten radiotårn
    \end{Verbatim}

    \begin{center}
    \adjustimage{max size={0.9\linewidth}{0.9\paperheight}}{Oving3-revidert-v1_files/Oving3-revidert-v1_87_1.png}
    \end{center}
    { \hspace*{\fill} \\}
    
    \begin{Verbatim}[commandchars=\\\{\}]
Histogram uten radiotårn
    \end{Verbatim}

    \begin{center}
    \adjustimage{max size={0.9\linewidth}{0.9\paperheight}}{Oving3-revidert-v1_files/Oving3-revidert-v1_87_3.png}
    \end{center}
    { \hspace*{\fill} \\}
    
    Her kan en se at nettverket fortsatt er veldig knyttet og histogrammet
sier fortsatt at alle noder har \textgreater2 kanter, som igjen betyr at
en alltid har en failsafe om en node skulle feile.

Dette kan jo igjen bety at nettverket er mer utsatt for angrep siden en
har færre noder. Så vi kan simulere et angrep der en ondsinned aktør
planlegger å deaktivere de to mest sentrale satelittene i nettverket.
Dette kan blant annet forekomme ved at en injiserer satelitten med
dårlig pakker som overskriver noe intern kode. Eller å produsere et
såkalt Distributed Denial of Service attack(DDoS) som overbelaster
nodene med trafikk.

    \begin{tcolorbox}[breakable, size=fbox, boxrule=1pt, pad at break*=1mm,colback=cellbackground, colframe=cellborder]
\prompt{In}{incolor}{179}{\boxspacing}
\begin{Verbatim}[commandchars=\\\{\}]
\PY{n}{vdes}\PY{o}{.}\PY{n}{remove\PYZus{}node}\PY{p}{(}\PY{l+s+s2}{\PYZdq{}}\PY{l+s+s2}{satilite0}\PY{l+s+s2}{\PYZdq{}}\PY{p}{)}
\PY{n}{vdes}\PY{o}{.}\PY{n}{remove\PYZus{}node}\PY{p}{(}\PY{l+s+s2}{\PYZdq{}}\PY{l+s+s2}{satilite1}\PY{l+s+s2}{\PYZdq{}}\PY{p}{)}
\PY{n}{vdes}\PY{o}{.}\PY{n}{draw}\PY{p}{(}\PY{p}{)}
\end{Verbatim}
\end{tcolorbox}

    \begin{center}
    \adjustimage{max size={0.9\linewidth}{0.9\paperheight}}{Oving3-revidert-v1_files/Oving3-revidert-v1_89_0.png}
    \end{center}
    { \hspace*{\fill} \\}
    
    Her har vi da fått en ring-graf med en båt liggende utenfor rekkevidde
av kommunikasjon. Dette er selvfølgelig katastrofalt for båten som nå
ikke har noen form for navigasjon eller informasjon fra utsiden. Men
rent grafisk er det et lovende resultat. Å miste de to mest sentrale
satelittene, og fortsatt ha kommunkasjon med 83.3\% av nodene er svært
robust.

    \begin{tcolorbox}[breakable, size=fbox, boxrule=1pt, pad at break*=1mm,colback=cellbackground, colframe=cellborder]
\prompt{In}{incolor}{180}{\boxspacing}
\begin{Verbatim}[commandchars=\\\{\}]
\PY{n}{vd} \PY{o}{=} \PY{n}{VDESGraph}\PY{p}{(}\PY{n}{boat\PYZus{}count}\PY{p}{,}\PY{n}{satellite\PYZus{}count}\PY{p}{,}\PY{n}{radio\PYZus{}tower\PYZus{}count}\PY{p}{)}
\PY{n}{vd}\PY{o}{.}\PY{n}{remove\PYZus{}node}\PY{p}{(}\PY{l+s+s2}{\PYZdq{}}\PY{l+s+s2}{satilite0}\PY{l+s+s2}{\PYZdq{}}\PY{p}{)}
\PY{n}{vd}\PY{o}{.}\PY{n}{remove\PYZus{}node}\PY{p}{(}\PY{l+s+s2}{\PYZdq{}}\PY{l+s+s2}{satilite1}\PY{l+s+s2}{\PYZdq{}}\PY{p}{)}
\PY{n}{vd}\PY{o}{.}\PY{n}{draw}\PY{p}{(}\PY{p}{)}
\end{Verbatim}
\end{tcolorbox}

    \begin{center}
    \adjustimage{max size={0.9\linewidth}{0.9\paperheight}}{Oving3-revidert-v1_files/Oving3-revidert-v1_91_0.png}
    \end{center}
    { \hspace*{\fill} \\}
    
    Hvis vi setter i verk det tiltaket med å redusere bruken av radiotårn
ville det være et robust system, der ``boat4'' fortsatt kunne
kommunisere nød til land og dermed være trygg.

Et slikt nettverk vil kunne være svært robust mot sannsynlige angrep mot
den fysiske infrastrukturen til VDES. At en da faser ut av slike trusler
betyr også at digitale angrep blir mer sannsynlige, så en må derfor være
bevisst på å sørge for at spesielt satelitter har sterke brannmurer mot
dårlig pakker, og at de tåler belastningen som tårnene tok seg av.


    % Add a bibliography block to the postdoc
    
    
    
\end{document}
